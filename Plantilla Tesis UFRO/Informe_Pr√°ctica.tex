% Definiciones de AMS-LaTeX: UFRO Thesis  Ramiro Donoso, Marzo 2017 *
% ------------------------------------------------------------------------
% Paquetes por instalar: srcltx, algorithms, algorithmic, algorithm e idioma espa�ol.
\documentclass[12pt,notitlepage]{report}
\usepackage[spanish]{babel}		% Silabeo en espa�ol. (corte de palabras!)
\usepackage[lighttt]{lmodern}
%\usepackage[latin1]{inputenc} 	% Caracteres con acentos.
\usepackage{cite}
\usepackage{graphicx}			% Para usar figuras JPG, PNG, etc...
\usepackage[centertags]{amsmath}
\usepackage{amsfonts}
\usepackage{amssymb} %letras matematicas (\mathbb{R})
\usepackage{amsthm}
\usepackage{algorithm}
\usepackage{algorithmic}
\usepackage{newlfont}
\usepackage[inactive]{srcltx}  %SRC Specials for DVI search
\usepackage{booktabs}
\usepackage{nonfloat} % para usar tabcaption y figcaption.... tablas no flotantes.
\usepackage{multirow} % para unir filas
\usepackage{multicol} % para unir columnas
\usepackage{dcolumn}
\usepackage{subfigure} % permite la utilizacion de subfiguras en una misma figura
\usepackage[subfigure]{tocloft}
\usepackage{longtable} % para cortar tablas en mas de una pagina
%\usepackage[justification=centering]{caption} % para caption largos y centrados.
\usepackage[final]{pdfpages}
%\usepackage{verbatim}

% ESTILOS PROPIOS
\usepackage{UFRO_Style} % Estilos propios de tablas, figuras y comandos.		
\usepackage{UFRO_Tesis}	% Incorpora el uso del paquete UFRO_Tesis
%\usepackage[noLinks]{UFRO_Tesis}	% Activa/Desactiva links en el documento.


\overfullrule=2mm
\hfuzz=2000pt


%%% INICIO DE DOCUMENTO 
\begin{document}
\hypersetup{%
 pdfborder = {0 0 0} 
 }
\ingCivTesis

\dept{Ingeniería Matemática}
\faculty{Ingeniería y Ciencias}
\carrera{Ingeniero Civil Matemático}
\title{Aumento de resolución de datos satelitales espacio-temporales a través del modelo Random Forest para el estudio de aguas subterráneas en Chile.}
\titlefoot{Aumento de resolución  de datos satelitales a través del modelo Random Forest}
\author{Matías Alfonso Palma Manterola}
\profguia{Dr. Walter Gómez Bofill}
\profcorrA{Ing. Leonardo Villegas Morales}
\profcorrB{Ing. Alejandro Ferreira Vergara}

% ------------------------------------------------------------------------

\ack{capitulo_0/Agradecimientos}  
\resumenesp{capitulo_0/Resumen}
\resumening{capitulo_0/Abstract}

\makeTitle{}
% ------------------------------------------------------------------------

\chapter{Introducción}

\section{Contexto y descripción del problema}
Las aguas subterráneas tienen una relación muy cercana con la sociedad humana gracias a que almacenan 
agua de buena calidad en grandes cantidades, se encuentran ampliamente distribuidas, poseen estabilidad 
y facilidad de uso por lo que se ha convertido en una fuente de suministro ideal y está siendo altamente 
utilizada por las personas \cite{11}, mayoritariamente por industrias de manera contínua y persistente sin 
fiscalización alguna. La sobreexplotación de los almacenes de agua subterránea ha generado que su nivel 
continúe disminuyendo, provocando desastres como hundimientos de suelo, pérdida de volumen a través de la
separación de un acuífero y el secado de zonas húmedas.
Las cuencas son monitoreadas en terreno a través de sondas, este estudio además de ser costoso requiere 
una distribución uniforme lo suficientemente distanciada entre cada estación de medición para captar en
detalle los cambios de nivel de los acuíferos. 
Con el surgimiento de algunas nuevas tecnologías de observación de la tierra, 
se ha proporcionado un nuevo método de detección remota para el estudio de las variaciones 
del almacenamiento de agua a nivel espacial, éste es el satélite GRACE, el cual a través de 
diferencias gravitacionales logra estimar la cantidad de agua presente en un punto geográfico. 
Muchos investigadores han demostrado que el satélite GRACE puede monitorear bien los cambios en áreas 
a gran escala con un error mínimo \cite{11}. Debido a que el satélite proporciona datos a una baja resolución es 
necesario realizar procesamiento de datos llamado reducción de escala que convierte datos de gran escala
y baja resolución en datos de pequeña escala a una alta resolución. 
%
%
%
%
\section{Satélite GRACE}
La misón GRACE (por sus siglas en inglés \textit{Gravity Recovery and Climate Experiment}), lanzada el 17 de marzo del año 2002
bajo el Programa \textit{Pathfinder} de Ciencias del Sistema Terrestre de la NASA \cite{11}, consiste en dos satélites 
que orbitan sobre la Tierra a una altitud aproximadamente de 500 Km y a una distancia entre ellos de 200 Km \cite{tws} para así detectar variaciones gravitacionales que resultan
de los movimientos de las masas, en particular de agua en cualquiera de sus estados.
Gracias a este par de satélites es posible detectar las variaciones de la fuerza de gravedad 
mediante el traspaso de información del primer satélite hacia el segundo, captando la diferencia de 
masas que existen en el terreno. Debido a estas diferencias gravitatorias se provoca un aumento o 
disminución de la distancia relativa entre un satélite y el otro logrando captar diferencias de hasta 
$1\mu m$, estas mediciones se conocen como \textit{TerrestriaL Water Storage(TWS)}
El gran objetivo de GRACE, es poder captar el movimiento de las aguas alrededor del globo para 
contrarrestar riesgos como sequías, inundaciones, desprendimiento de suelos o socavones.

\subsection{Almacenamiento de agua terrestre}
El Almacenamiento de agua terrestre o TWS (por sus siglas en inglés \textit{Terresttial Water Storage}), es definido como una estimación integrada del agua almacenada sobre y debajo de la superficie terrestre, 
incluyendo el agua de dosel(\textit{canopy water}), ríos y lagos, humedad de suelo(\textit{soil moisture}) y aguas subterráneas \cite{tws}. Es calculado de la siguiente manera:
\begin{equation}
    \text{TWS Anomaly}_t = \frac{TWS_t-\bar{X}}{\delta}
\end{equation}
donde $TWS_t$ es el valor de TWS en el mes $t$, $\bar{X}$ es el valor promedio a largo plazo de TWS y $\delta$ es su desviación estándar. Ambas calculadas para el mismo mes $t$ utilizando 
los datos temporales disponibles \cite{tws}.

En base a las anomalías se pueden definir ciertos rangos de comportamientos de las mediciones que se detallan en la figura \ref{logo1}, en ella se puede observar que para la anomalía TWS, si ésta se encuentra en el intervalo $[-2,-1[$
implica que  el contenido terrestre de agua está más seco de lo normal. Por el contrario si la observación se encuentra en el intervalo $[1,-1[$ representa que el contenido de agua ha aumentado y finalmente si se encuentra entre
$[-1,1[$ representa ``normalidad'' en el contenido de agua, es decir, la cantidad de agua presente no ha variado considerablemente.

\begin{figure}[H]
    \centering
          \subfigure[Zonas con estrés hídrico en base a TWS.]{\includegraphics[scale=0.07]{images/tws-example.png}\label{fig:tws}}
          \subfigure[Descripción general del intervalo donde oscila TWS.]{\includegraphics[scale=0.1]{images/tws-range.png }\label{fig:tws1}}\goodgap
          \vskip -0.1in
    \caption[Ejemplo de anomalías de TWS]{\footnotesize Ejemplo de anomalías de TWS donde se destacan las condiciones de estrés hídrico en Australia 2019.}
    \label{logo1}
\end{figure}
Este indicador juega un  papel crucial en el ciclo hidrológico mundial y el proceso de integración tierra-atmósfera. Las variaciones en \textit{TWS} son
un fuerte indicador del equilibrio o desequilibrio de los flujos de agua afectados por las condiciones climáticas o geográficas
regionales y es uno de los factores importantes que vinculan la circulación general de la atmósfera con los desastres naturales,por ejemplo, la sequía \cite{15}.
%
%
%
%
\section{(CR)$^2$}
El Centro de Ciencia del Clima y la Resiliencia (CR)$^2$ es un centro de excelencia cuyo propósito es generar investigación
sobre la ciencia del clima y la resilencia desde un enfoque interdisciplinario. 

(CR)$^2$ nace en 2013 financiado por Fondap de la Agencia Nacional de Investigación y Desarrollo (ANID). En 2018 comienza una segunda etapa
de investigación donde busca consolidarse como un actor clave para la ciencia del clima y así contribuir al tránsito del país hacia un desarrollo 
bajo en carbono \cite{cr2}.

%
%
%
%
\section{Dirección General de Aguas}
La Dirección General de Aguas es el organismo del Estado de Chile que se encarga de gestionar, verificar y difundir la infromación 
hídrica del país, en especial, respecto a su cantidad y calidad, las personas naturales o jurídicas que están autorizadas a utilizarlas,
las obras hidráulicas existentes y la seguridad de las mismas con el objetivo de contribuir a una mayor competitividad 
del mercado y el resguardo de la certeza jurídica e hídrica para el desarrollo sustentable del país \cite{dga}.

%
%
%
%
\section{Climate Data Operator}
El software \textit{Climate Data Operator(CDO)} es una colección de operadores para procesamiento y predicción estándar de datos climatológicos.
Los operadores incluyen funciones estadísticas y aritméticas simples, selección de datos, herramientas de submuestreo e interpolación espacial.
\textit{CDO} fue desarrollado para tener el mismo conjunto de funciones de procesamiento para archivos con extensión GRIB y NetCDF. Las mayores características
de CDO son, entre otras, los más de 700 operadores disponibles, su diseño modular y fácilmente extensible a nuevos operadores, la interfaz de linea de comando UNIX muy simple,
el conjunto de datos en cuestión puede ser procesado mediante varios operadores sin la necesidad de almacenar los resultados provisionales en archivos y finalmente, un rápido procesamiento
de conjuntos de datos muy grandes. \cite{16}.
%\newpage
%
%
%
%
\section{Objetivo General}

Implementar \textit{Downscalling} a los datos satelitales GRACE a través de un modelo predictivo para obtener 
productos grillados de \textit{Terrestrial Water Storage} en Chile a una alta resolución.
%
%
%
%
\section{Objetivos Específicos}
\begin{itemize}
    \item Definir la metodología del \textit{Downscalling}.
    \item Adquirir conocimientos acerca de tratamiento de datos satelitales georreferenciados utilizando Python.
    \item Recolectar datos necesarios para implementar el aumento de resolución en el territorio.
    \item Analizar y Pre-procesar los datos. 
    \item Implementar el modelo predictivo y medir su rendimiento en base a métricas definidas.
    \item Validar las predicciones del modelo en base a observaciones \textit{in situ} a través de series de tiempo.
\end{itemize}

\chapter{Fundamentos Teóricos}
\label{C2}
\section{Reducción de escala}
La reducción de escala (en inglés, \textit{Downscaling}) es un procedimiento utilizado para inferir 
información de alta resolución, como por ejemplo datos satelitales, a partir de variables de baja 
resolución. Esta técnica se basa en enfoques dinámicos o estadísticos comúnmente utilizados en varias 
disciplinas, especialmente la meteorología, la climatología y la teledetección. El término reducción 
de escala generalmente se refiere a un aumento en la resolución espacial, pero a menudo se usa para la 
resolución temporal.
%
%
%
%
\section{Datos georreferenciados}
La georreferenciación es el uso de coordenadas de un mapa para asignar una ubicación espacial a entidades
cartográficas. Los elementos del conjunto de datos tienen asignados una ubicación geográfica y una extensión específicas
que permiten situarlos en la superficie terrestre o cerca de ella.

La correcta descripción de la ubicación requiere un marco para definir ubicaciones del mundo real. Un sistema de coordenadas geográficas
se utiliza para asignar ubicaciones georreferenciadas a los datos. El sistema de coordenadas latitud-longitud global es uno de los marcos.
Otro marco muy utilizado resulta ser el sistema de coordenadas cartesianas que surge a partir del marco global a través de ciertas transformaciones
en el cuerpo de los números complejos ($\mathbb{C}$).

La ventaja de trabajar con datos que contengan georreferencias, es que con estos pueden generarse grillas cuadriculadas regulares, es decir, mediante una distancia arbitraria (generalmente determinada en grados $[^{\circ}]$)
es posible generar observaciones equidistantes entre sí lo cual entrega regularidad para remuestrear y analizar datos.

    \subsection{Longitud y Latitud}
    Un método para describir la posición de una observación en alguna de las superficies (comúnmente llamadas \textit{capas}) consiste en utilizar 
    mediciones esféricas de latitud y longitud. Estas son mediciones de los ángulos (en grados) desde el centro de la Tierra hasta un punto de una determinada
    capa. Este tipo de coordenadas se denominan coordenadas geográficas.

    La \textbf{longitud} mide ángulos en una dirección este-oeste. Las mediciones de longitud usualmente se basan en el meridiano de Greenwich, que es una línea
    imaginaria que realiza un recorrido desde el Polo Norte hasta el Polo Sur, en este meridiano la longitud es 0$^{\circ}$. El oeste del meridiano de Greenwich por lo general 
    se registra como longitud negativa y el este, como longitud positiva.

    La \textbf{latitud} mide ángulos en una dirección norte-sur. Las mediciones de latitud comienzan desde la línea del Ecuador, que es la máxima circunferencia perpendicular
    al eje de rotación del planeta Tierra, en este paralelo la latitud es 0$^{\circ}$. El norte de la línea ecuatorial por lo general 
    se describe como latitud positiva y al sur, como latitud negativa.

    Si bien la longitud y latitud se pueden ubicar en posiciones exactas de la superficie de la Tierra, estas no proporcionan
    unidades de medicion uniformes de longitud y latitud. Solo a lo largo del ecuador la distancia que representa un grado de longitud se aproxima a la distancia
    en un grado de latitud. Esto se debe a que el ecuador es el único paralelo tan extenso como cualquier meridiano, ya que si nos alejamos del ecuador, el tamaño circunferencial de
    los paralelos disminuye hasta transformarse en un punto en los polos donde los meridianos convergen.
%    
%
%
%
\section{Error}
En modelos predictivos, la diferencia entre un valor real y una estimación, o aproximación se conoce como error, por lo tanto, mientras más pequeño sea
la estimación se vuelve más verídica. En otras palabras, el error es utilizado para medir el rendimiento de un modelo predictivo y así poder
realizar comparaciones entre diferentes estimadores con el fin de seleccionar el mejor.
Hay muchas maneras de medir el error de acuerdo a las necesidades que se dispongan.


En un modelo de clasificación el error y la exactitud van de la mano, ya que el primero se puede definir como el porcentaje de clasificaciones erradas que realiza el modelo 
y la exactitud es el porcentaje de aciertos de clasificación.
En cambio, en un modelo regresivo no podemos establecer una relacion tan directa ya que la predicción se encuentra en un dominio continuo, por lo tanto, si se mide el 
error de la misma forma que para un modelo de clasificación, la mayoría de las veces entregará un valor muy cercano al 100\%, es por esto que se recurre a otras maneras de medirlo: 
el \textbf{error medio absoluto} y el \textbf{error cuadrático medio}, en inglés \textit{mean absolute error (MAE)} y \textit{mean squared error (MSE)}, respectivamente.

    
    \subsection{Error medio absoluto}
    El error medio absoluto, o \textit{MAE}, es una métrica de evaluación utilizada
    en modelos regresivos. El error medio absoluto de un modelo respecto a un conjunto de prueba (o \textit{test}) resulta ser, como su nombre lo indica, la media de los
    valores absolutos de los errores de predicción individuales en todas las instancias del conjunto \textit{test} \cite{5}.
    
    \begin{equation}\label{eqn:mae}
        mae = \frac{1}{n} \sum_{i=1}^n|~y_i - \bar{y}_i~|
    \end{equation}

    En la ecuación \ref{eqn:mae}, $y_i$ es el valor real de la variable objetivo de la instancia $i$, $\bar{y}_i$ es la predicción de la $i$-ésima instancia en el mismo conjunto
    y $n$ es la cantidad de elementos en el conjunto de prueba.

    \subsection{Error cuadrático medio}
    El error cuadrático medio, o \textit{MSE}, es una métrica de evaluación utilizada
    en modelos regresivos. El error cuadrático medio de un modelo respecto al conjunto de prueba es la media del cuadrado de los errores de predicción 
    en cada instancia del conjunto \cite{4}.

    \begin{equation}\label{eqn:mse}
        mse = \frac{1}{n} \sum_{i=1}^n(y_i - \bar{y}_i)^2
    \end{equation}

    En \ref{eqn:mse}, $y_i$ es el valor real de la variable objetivo de la instancia $i$, $\bar{y}_i$ es la predicción de la $i$-ésima instancia en el mismo conjunto
    y $n$ es la cantidad de elementos en el conjunto de prueba.
%   
%
%
%
\section{Bootstrap}
En el contexto de estadísticas y ciencia de datos, el \textit{bootstrapping} es un método para inferir resultados
de una población a partir de resultados obtenidos de una colección aleatoria igual o más pequeña de esa población. Por tanto, 
el método requiere la noción de \textit{bootstrap sample}, la cual se define como una \textbf{muestra aleatoria con reposición}
\cite{17}. Esto significa que, luego de que una observación de los datos ha sido elegida del conjunto para obtener la muestra, este sigue 
estando disponible para una selección posterior. La \textit{muestra de bootstrap} debe ser del mismo tamaño que el conjunto de datos original \cite{18}, debido a esto, 
algunas muestras serán consideradas múltiples veces mientras que otras no serán seleccionadas. Las muestras que no han sido seleccionadas 
se conocen como \textit{out of bag samples}, o muestras fuera de la bolsa, ya que estas no se incluirán dentro del modelo.

%
%
%
%
\section{Métodos basados en árboles}

Los métodos basados en árboles son algoritmos de \textit{Machine Learning} del tipo supervisado donde la  
clasificación o predicción se puede ver como una división de los datos en un conjunto de rectángulos \cite{13} y en
cada uno se ajusta un modelo tomando una decisión de acuerdo a una serie de consultas, el modelo
aprende la respuesta a una serie de interrogantes para inferir la etiqueta de clase de los
ejemplos (clasificación) o también la predicción de números reales (regresión). En otras palabras, 
Los modelos basados en árboles involucran \textit{estratificación} o \textit{segmentación} del 
espacio de predicción dentro de un número de regiones simples, donde la predicción se lleva a cabo 
en base a la media o la moda de las observaciones de entrenamiento de la región a la que pertenece \cite{14}, según las necesidades.

Los métodos basados en reglas y los métodos basados en árboles son herramientas populares debido a varias razones,
una de ellas es debido a su construcción lógica \cite{18}, la cual permite manejar eficientemente muchos tipos de predictores
, por ejemplo predictores continuos (regresión) o categóricos sin la necesidad de procesar datos. Además,
estos modelos logran lidiar de forma eficiente los valores faltantes e implicitamente llevan a cabo una selección de características,
lo cual resulta importante para problemas de modelación de la vida real.

Estos modelos poseen ciertas falencias, dos de ellas en particular bien estudiadas:
\newpage
\begin{enumerate}
    \item La \textbf{inestabiliad} del modelo, es decir, ligeros cambios en las variables explicativas 
          pueden llevar a grandes cambios en la estructura del árbol.
    \item El \textbf{rendimiento predictivo menor} que el optimal, esto debido a que el modelo define estructuras rectangulares y si 
          la relación entre las variables explicativas y la variable respuesta no puede ser definida adecuadamente por
          subespacios rectangulares de los predictores, entonces el modelo tendrá un mayor error de predicción.
\end{enumerate} 


Para lidiar con estos inconvenientes, investigadores desarrollaron conjuntos de modelos que combinan una cantidad fija de árboles
y una variación del \textit{Bootstrap} para mejorar el rendimiento.

    \subsection{Árboles de Regresión}
    Existen muchas formas de construir árboles de regresión, una de las más antiguas y más utilizadas
    resulta ser el árbol de clasificación y regresión \textit{CART}, por sus siglas en inglés 
    \textit{Clasification And Regresion Tree}, el cual comienza buscando todos los posibles valores de cada predictor para poder encontrar a cada uno,
    o en otras palabras, poder conocer su comportamiento y determinar el valor que divide los datos en dos subconjuntos, de modo que la suma de errores 
    cuadrados de cada subconjunto sea minimizada \cite{18}. Entonces, si denotamos como $X$ el conjunto de todos los datos y como $X_1 \text{ y } X_2$ a la
    partición de este mediante \textit{CART}, entonces el error que se busca minimizar es la suma de \textit{MSE} (ec. \ref{eqn:mse}) en cada partición, esto es:
    \begin{equation*} 
        SSE = \sum_{i\in X_1}(y_i-\bar{y}_1)^2+ \sum_{j\in X_2}(y_j-\bar{y}_2)^2
    \end{equation*}
    donde $y_i$ y $y_j$ representan la predicción del modelo y $\bar{y}$ representa la media de la predicción en cada subconjunto.

    Este método utiliza una división a partir del conjunto entero de datos aumentando la profundidad del modelo. En la práctica el proceso continúa
    dentro de los conjuntos $X_1$ y $X_2$ hasta que el número de divisiones en la muestra cae debajo de cierto umbral, ahí culmina el proceso de
    \textit{paso de crecimiento del árbol} \cite{18} o también llamada profundidad del árbol.
    Por esta razón este método es comúnmente conocido como \textbf{partición recursiva}. Sin duda el mayor de los problemas es debido a la inestabilidad que conduce a una alta varianza. 
    A menudo un pequeño cambio en los datos podría resultar en una serie de divisiones muy diferente, precarizando la predicción. La mayor razón de inestabilidad es la naturaleza 
    jerárquica del proceso; El efecto de un pequeño error en una de las primeras divisiones es propagada hacia abajo por todas las divisiones siguientes \cite{13}.
%
%
%
%
\section{Aprendizaje de conjunto}
En los años 90's comenzaron a emerger métodos que combinan las predicciones de muchos modelos para generar una nueva, estas se llamaron \textbf{técnicas de conjunto},
o en inglés \textit{ensemble techniques} \cite{18}.

El aprendizaje en conjunto (\textit{ensemble learning}) se refiere a los procedimientos empleados para entrenar múltiples máquinas de aprendizaje y combinar sus resultados, tratándolos como un ``comité"$~$de tomadores de decisiones. 
El principio es que la decisión del comité, con las predicciones individuales combinadas adecuadamente, debería tener una mejor precisión general, en promedio, que cualquier miembro individual del comité. 
Numerosos estudios empíricos y teóricos han demostrado que los modelos de conjuntos muy a menudo alcanzan una mayor precisión que los modelos individuales.
Los miembros del conjunto pueden estar prediciendo números con valores reales, etiquetas de clase, probabilidades, clasificaciones, agrupaciones o cualquier otra cantidad. Por lo tanto, 
sus decisiones se pueden combinar mediante muchos métodos, incluidos los métodos de promediación, votación y probabilísticos. La mayoría de los métodos de aprendizaje por conjuntos son genéricos, 
aplicables en varios tipos de modelos y tareas de aprendizaje automático \cite{6}.


\section{Bagging}
\textit{Bagging}, es una palabra compuesta por los términos en inglés \textit{\textbf{B}ootstrap \textbf{agg}regat\textbf{ing}}, 
debido a que consite en un método que utiliza el \textit{bootstrap} en conjunto con cualquier regresión o clasificación
para generar múltiples versiones de un predictor y usarlo para obtener un predictor ``agregado'' \cite{22}; Cada modelo en el conjunto es utilizado para generar una predicción de una observación en particular 
y luego se promedian entre ellas para obtener una predicción \textit{bagged} del modelo .
Este método fue introducido y propuesto por Leo Breiman y es una de las técnicas
de conjunto más nuevas. El pseudocódigo de este método es el siguiente:

\begin{algorithm}[H]
    \caption{Bagging}

    \begin{algorithmic}[1]
        \FOR{$i=1\text{ to }m$} 
        \STATE{Generar una muestra de \textit{bootstrap} de los datos originales}
        \STATE{Entrenar el modelo en la muestra generada en 2}
        \ENDFOR

    \end{algorithmic}
\end{algorithm}

\textit{Bagging} funciona mejor con modelos que sean inestables, es decir, aquellos que producen diferentes
patrones de generalización con pequeños cambios a los datos de entrenamiento. Por lo tanto, el \textit{bagging} no tiende a funcionar bien con modelos lineales \cite{19}.

Una de las ventajas de esta metodología es la reducción de varianza de la predicción a través del processo de agregación. Además, para aquellos modelos que producen una predicción inestable,
o de alta varianza como lo son los árboles de regresión hace que estos sean más estables \cite{18} y por tanto, más precisos.

La desventaja más clara del método resulta ser el alto costo computacional, ya que este aumentará a medida que el número de muestras
de \textit{bootstrap} aumenten. Este problema es solucionable si se posee acceso a cómputo paralelo debido a que el \textit{bagging} es fácilmente paralelizable \cite{18}.
%
%
%
%
\section{Modelo del bosque aleatorio}
Los bosques aleatorios son un \textit{ensemble} de predictores de árboles de modo que cada árbol depende de los valores de un vector aleatorio muestreado de forma independiente (\textit{bootstrap sample}) y con la misma distribución para cada uno de ellos. 
El error de generalización para los bosques converge estable asintóticamente a un límite a medida que aumenta el número de árboles en el bosque, además, depende entre otras cosas, de la correlación entre ellos. Usar una selección aleatoria de características para dividir cada nodo produce
tasas de error que se comparan favorablemente con otros modelos como \textit{Adaboost}, pero son más sólidas con respecto al ruido, error de estimaciones internas,
fuerza y correlación, y se utilizan para mostrar la respuesta al aumento del número de características utilizadas
en la división. Las estimaciones internas también se utilizan para medir la importancia de la variable \cite{25}.

La ventaja que más resalta del modelo del bosque aleatorio recae en que es un modelo \textbf{escalable} y además fácil de usar. La idea detrás es promediar múltiples, además de profundos, 
árboles de decisión los cuales de forma individual sufren de una alta varianza. Utilizando este conjunto de árboles es posible obtener una mejor generalización en la predicción y además de ser mucho menos
susceptible al sobreajuste, es decir, ajustarse de sobremanera al conjunto de prueba \cite{24}.

El modelo en cuestión se resume en el siguiente pseudocódigo:

\newcommand{\INDSTATE}[1][1]{\STATE\hspace{#1\algorithmicindent}}

\begin{algorithm}[h]
    \caption{Modelo Random Forest}

    \begin{algorithmic}
            \STATE 1. \textbf{Construír un \textit{bootstrap sample}} de tamaño $n$.
            \STATE 2. \textbf{Generar un árbol de clasificación y regresión} para la muestra de \textit{bootstrap} y en cada nodo:

            \INDSTATE a) Aleatoriamente \textbf{seleccionar $d$ características} sin reposición.

            \INDSTATE b) \textbf{Dividir el nodo} utilizando las características que proporcionan la mejor división de\INDSTATE acuerdo con la función objetivo(maximización de la ganancia de información, minimi-\INDSTATE zación del \textit{RMSE}, entre otras).
            
            \STATE 3. \textbf{Repetir} los pasos 1. y 2. $k$ veces.
            \STATE 4. \textbf{Agregar la predicción de cada uno de los $k$ árboles} para obtener una predicción \textit{bagged}, es decir, una predicción que combina de alguna forma, por ejemplo el promedio, 
            de las predicciones de cada árbol por sí solo.
    \end{algorithmic}
\end{algorithm}



%
%
%
%
\section{Interpolación bilinear}
La interpolación bilinear es una extensión de la interpolación lineal en una variable con el objetivo de interpolar funciones de dos variables en una malla regular, por ejemplo coordenadas geográficas o imágenes.
La idea principal es realizar una interpolación lineal en una dirección y luego en la otra. Aunque las operaciones en cada dirección son lineales, la interpolación bilinear es no lineal. 

Si se tiene la observación en los 4 vecinos más cercanos de la grilla a un determinado punto $(n_1, n_2)$, digamos 
$f(n_{10}$,$n_{20})$, $f(n_{11},n_{21})$, $f(n_{12},n_{22})$, $f(n_{13},n_{23})$, entonces en la coordenada $(n_1, n_2)$ se define la 
función bilinear $g(n_1, n_2)$ que aproxima la observación como
\begin{equation}
    g(n_1, n_2) = A_0 + A_1 n_1 + A_2n_2 + A_3n_1n_2
\end{equation}

Los llamados ``pesos bilineares''\cite{26} $A_0,~A_1,~A_2\text{ y }A_3$, son calculados resolviendo el siguiente sistema de ecuaciones:
\begin{equation}\label{interp2d}
    \begin{bmatrix}
        A_0 \\
        A_1 \\
        A_2 \\
        A_3 \\
    \end{bmatrix}
    =
    \begin{bmatrix}
        1 & n_{10} & n_{20} & n_{10}n_{20} \\
        1 & n_{11} & n_{21} & n_{11}n_{21} \\
        1 & n_{12} & n_{22} & n_{12}n_{22} \\
        1 & n_{13} & n_{23} & n_{13}n_{23} 
    \end{bmatrix}^{-1}
    \begin{bmatrix}
        f(n_{10},n_{20}) \\
        f(n_{11},n_{21}) \\
        f(n_{12},n_{22}) \\
        f(n_{13},n_{23}) \\
    \end{bmatrix}
\end{equation}
Por lo tanto $g(n_1,n_2)$ está definido para ser una combinación linear de los 4 vecinos más cercanos de una determinada observación.
%
%
%
%
\section{Series de tiempo}
Una serie de tiempo consite en un conjunto de observaciones $x_t$, cada una es registrada en un tiempo fijo $t$.
Una serie de tiempo a \textbf{tiempo discreto} es cuando el intervalo de tiempos $T_0$ en donde se llevan a cabo las observaciones
es discreto, por ejemplo, cuando se realizan en intervalos de tiempos fijos (mediciones diarias, mensuales, anuales, etc.). 
En cambio, las series de tiempo a \textbf{tiempo continuo} son obtenidas cuando las observaciones recopiladas se encuentran medidas sobre un intervalo
contínuo de tiempo, por ejemplo $T_0=[0,1]$ \cite{21}. Una serie de tiempo representa la relación entre dos variables, el tiempo es la primera 
y la segunda es cualquier variable cuantitativa. No es necesario que la relación muestre siempre incremento en el cambio de la variable con referencia al tiempo. 

Los usos más comunes para realizar estudios de series de tiempo son variados, entre ellos se encuentran la predicción de comportamientos
futuros de la variable cuantitativa basada en las observaciones o registro histórico, la planificación de negocios en base a comparación del 
rendimiento real con el esperado, comparar comportamientos entre variables para cierto lugar geográfico, entre muchas otras. 


    \subsection{Componentes de las series de tiempo}
    
    Las diversas razones o fuerzas que afectan los valores de una observación en una serie de tiempo son sus componentes. 
    Las cuatro categorías de las componentes de las series de tiempo son la \textbf{tendencia}, la \textbf{variación estacional}, la \textbf{variación cíclica} 
    y los \textbf{movimientos irregulares} o también llamados residuos.

    \subsubsection{Tendencia}
    Muestra la tendencia general de los datos a aumentar o disminuir durante un largo período de tiempo. Una tendencia es de media suave, general y 
    a largo plazo. No siempre es necesario que el aumento o la disminución sea en la misma dirección a lo largo del período de tiempo dado.
    Se observa que las tendencias pueden aumentar, disminuir o mantenerse estables en diferentes tramos de tiempo. Pero la tendencia general debe ser al alza, a la baja o estable. 
    Usualmente se refiere a un ``cambio de dirección en la tendencia'' cuando el componente varía de una tendencia incremental a una decreciente o viceversa \cite{20}.

    \subsubsection{Estacionalidad y Cíclica}
    
    La \textbf{estacionalidad} son las fuerzas rítmicas que operan de manera regular y periódica en un lapso de menos de un año. Tienen el mismo o casi el mismo patrón durante un período de 12 meses. 
    Además, siempre es obtenida a partir de un periodo fijo y conocido de tiempo \cite{20}, ya sea cada hora, día, semana, trimestre o mes.
    
    Estas variaciones entran en juego debido a las fuerzas naturales o por convenciones hechas por el hombre. Las diversas estaciones o condiciones climáticas juegan un papel 
    importante en las variaciones estacionales. 

    Las variaciones en una serie de tiempo que operan en un lapso de más de un año son las \textbf{variaciones cíclicas}. Este movimiento oscilatorio tiene un período de más de un año que se conoce como ciclo.
    Este movimiento a veces se denomina ``ciclo económico''.

    Es un ciclo de cuatro fases que comprende las fases de prosperidad, recesión, depresión y recuperación. La variación cíclica puede ser regular no periódica.

    \subsubsection{Residuos}
    
    Hay otro factor que provoca la variación en la variable en estudio. No son variaciones regulares y son puramente aleatorias o irregulares. 
    Estas fluctuaciones son imprevistas, incontrolables, impredecibles y erráticas. Estas fuerzas son terremotos, guerras, inundaciones, hambrunas y cualquier otro desastre.

    \subsection{Descomposición de series de tiempo}
    Los datos en series de tiempo pueden exhibir una gran variedad de patrones y es útil categorizarlos y estudiar
    comportamientos que se puedan apreciar en ellas. Otras veces es útil intentar dividir la serie de tiempo en varios componentes,
    cada uno representando uno de los patrones subyacentes \cite{20}.

    La forma más sencilla y popular de descomponer una serie de tiempo es pensar que la serie $y_t$ se encuentra compuesta por 
    3 componentes: la componente estacional, la componente tendencial-cíclica y la componente residual que resulta ser
    todo lo restante para completar la serie \cite{20}.

    Si se asume que la serie de tiempo se puede descomponer de forma aditiva, entonces:
    \begin{equation}
        y_t = S_t + T_t + E_t
     \end{equation}
        
     En la ecuación,  $y_t$ es la observación o variable cuantitativa en el instante $t$, $S_t$ es la componente estacional, $T_t$ es la componente tendencial cíclica
     y $E_t$ es la serie de tiempo residual o también llamado \textit{irregular} \cite{20}, todas ellas observadas en el mismo instante $t$.

     De manera alternativa, una descomposición multiplicativa de la serie de tiempo puede escribirse como:
     \begin{equation}
        y_t = S_t 
        \times T_t 
        \times E_t
     \end{equation}
     la cual es equivalente a una doscomposición logarítimca de la serie, ya que:
     \begin{equation*}
        y_t = S_t 
        \times T_t 
        \times E_t
        \iff
        \log y_t =  \log S_t +
        \log T_t +
        \log E_t
     \end{equation*}

     El modelo aditivo resulta más apropiado si la magnitud de las fluctuaciones estacionales o las variaciones
     a través de la componente tendencial-cíclica no varía con el nivel de la serie de tiempo.


     \subsection{Comparación de series de tiempo}
     En la práctica, es frecuente enfrentar situaciones
     en las que se requiere comparar dos o más series temporales, o analizar un gran
     número de estas series para separarlas en grupos tan homogéneos como sea posible \cite{27}.
     Normalmente, se generan bases de datos temporales que luego se estudian para identificar concordancias y disimilitudes.    

     Una de las medidas convencionales de proximidad es la distancia euclídea. Si bien esta medida es muy utilizada, ignora la interdependencia entre
     las mediciones de una serie temporal y basa la similitud entre dos series únicamente en la proximidad entre valores observados en los mismos instantes de
     tiempo \cite{27}.
     \subsubsection{Índice de correlación de Pearson}
     La correlación temporal, o índice de correlación de Pearson, es un índice que tiene por objetivo medir la similitud lineal en las tasas de crecimiento de dos series en períodos de
     tiempo de la forma $[t_i; t_{i+1}]$. Se define como:
     \begin{equation}\label{eqn:cort}
        CorrT(S_1,S_2) = \frac{\sum_{j=1}^p\sum_{i=1}^p(u_i-u_j)(v_i-v_j)} {\sqrt{\sum_{i=1}^p(u_i-u_j)^2}\sqrt{\sum_{i=1}^p(v_i-v_j)^2}}
     \end{equation}

     En la ecuación \ref{eqn:cort}, $S_1$ y $S_2$ son las series temporales a comparar las cuales poseen observaciones $ u_1,\dots,u_p $ y $ v_1,\dots,v_p$ respectivamente.

     El coeficiente de correlación temporal oscila en el intervalo $[-1,1]$. Si el coeficiente posee un valor de 1 quiere decir que en cualquier período de tiempo $[t_i,t_{i+1}]$ las series
     que se están comparando han aumentado o disminuído de manera simultánea con la misma tasa de creciemiento. En cambio, si alcanza el valor de -1 quiere decir que en cualquier periodo, si una de ellas ha
     aumentado, la otra ha disminuído o viceversa. Finalmente, si alcanza el valor de 0 significa que las tasas de crecimiento o decrecimiento son estocásticamente linealmente independientes \cite{27}.

     \subsubsection{Anomalías porcentuales}
     Una manera de poder realizar comparaciones en series de tiempo es mediante el cálculo de anomalías porcentuales. La gran ventaja que posee esta, es que el recorrido de la serie de tiempo
     se escala en valores porcentuales para determinar qué tanto varía una serie de tiempo respecto a su media. La fórmula para calcular las anomalías porcentuales es la siguiente:

    \begin{equation}\label{pct_anomalies}
        pct_t = 100 \cdot\frac{x_t-\mu}{\mu}
    \end{equation}
    Aquí, $x_t$ corresponde a cada componente temporal de la serie de tiempo y $\mu$ representa el promedio de toda la serie. 


\chapter{Metodología}
\label{C3}
Para cumplir el objetivo general se utilizó el lenguaje de programación Python debido a que es un lenguaje de \textbf{alto nivel}, por tanto es más fácil de utilizar como de leer debido a su sintaxis simple.
Además, Python resulta ser un lenguaje \textbf{polivalente multiparadigma} lo que hace que sea de propósito general, en particular es una excelente herramienta para \textit{Data Science}. Otro aspecto importante es la
\textbf{portabilidad}, la cual permite que el código no deba ser modificable de ninguna manera para ejecutarse en cualquier sistema operativo (macOS, Linux, UNIX y Windows) y finalmente que es un lenguaje
\textbf{gratis y de código abierto} lo que permite que pueda ser usado, modificado y distribuido por cualquier persona.

Tras un estudio exhaustivo de referencias y documentación se decide optar por aquella metodología
que se acercase más a la problemática local con datos que puedan ser obtenidos fácilmente. En base a esto se escoge el paper \textit{Downscaling of GRACE-Derived Groundwater Storage
Based on the Random Forest Model} \cite{11}. 

Otra de las herramientas utilizadas durante todo el proceso fue la Máquina Vritual (VM por sus siglas en inlés \textit{Virtual Machine}) creada en 
\textit{Google Cloud Plataform (gcp)} y accediendo a ella mediante una conexión \textit{ssh} a través del perfil de \textit{GitHub}. En pocas palabras el proceso consiste, en primer lugar, crear las máquinas virtuales;
En este caso se generaron 2 VM con diferentes características y funciones, la primera está destinada para el procesamiento más extenso y que necesite una alta capacidad de procesamiento para llevar a cabo las tareas más complejas en menos tiempo,
así, las características de la máquina se resumen en el código \texttt{e2-highmem-16} que se traduce como un hardware de una CPU de 16 núcleos virtuales de 128 GB de memoria RAM y la segunda VM es la que se utilizará
más constantemente con una capacidad menor pero que sea capaz de soportar el nivel de cálculo necesario, las carácterísticas de la máquina codifcadas son \texttt{e2-custom-6-21760} que consta de 6 núcleos virtuales,
2 físicos y aproximadamente 21 GB de RAM. Ambas máquinas virtuales utilizan un disco persistente balanceado de 100GB. 

Luego de haber generado los entornos en los cuales se procesarán y guardarán los datos, se crea el repositorio en \textit{GitHub} en 
donde se irá guardando y archivando el código generado para realizar el \textit{Downscaling}. La ventaja de utilizar \textit{git} es que proporciona un espacio seguro y con respaldo ante cualquier eventualidad.

\begin{figure}[H]\label{sshconec}
    \centering
          \includegraphics[width=0.6\linewidth]{images/ssh-conection.png}
    \caption[Conexión ssh de máquina virtual]{\footnotesize Conexión ssh de máquina virtual ``matias-exploracion-5'' (\texttt{e2-custom-6-21760}) con el \textit{e-mail} ``m.palma14'' asociado usuario de \textit{GitHub} mediante el archivo \textit{google-connect-docker.sh} en LINUX.}
\end{figure}

La tercera parte consiste en configurar los datos globales de nombre de usuario y \textit{e-mail} en \textit{git} para
enlazar a la conexión, posteriormente crear una llave de conexión \textit{ssh} y agregarla al repositorio creado en \textit{GitHub}, para lograr conectarse a cualquiera de las máquinas virtuales que se deseen utilizar y así poder realizar modificaciones 
al repositorio y registrarlas mediante \textit{commits}.
Luego de haber activado la máquina virtual en \textit{gcp} que se desea utilizar se procede a conectarla (fig. \ref{sshconec}.1) con la dirección IP del computador personal 
mediante un archivo con extensión \textit{.sh} el cual solicita la información del usuario y de la máquina y la coteja.

Para finalizar, lo único que resta es elegir el puerto al cual el usuario desea conectarse en el navegador ya que posee múltiples entornos de ejecución de código Python, entre ellos \textit{Jupyter Notebook}, \textit{Jupyter Lab} y \textit{Dask Dashboard}
los cuales entregan diferentes herramientas y formatos para realizar el Análisis de Datos. Para este trabajo se optó por utilizar \textit{Jupyter Lab} ya que es aquel que permite tener un control más detallado del directorio de archivos, acceso a la terminal
de la VM de forma rápida, lo cual aumenta su eficiencia a la hora de trabajar con la máquina remota. 


%
%
%
%
\section{Métodología propuesta en el artículo}

\begin{figure}[H]
    \centering
          \includegraphics[scale=0.7]{images/metodología_Chen_etal2019.png}
          \vskip -0.1in
    \caption[Metodología de reducción de dimensión propuesta]{\footnotesize Metodología propuesta en \cite{11}, pág. 7.}
    \label{metodologia}
\end{figure}

En base a \cite{11}, se procede a replicar la metodología propuesta en la figura 3.1, donde se detalla
la forma en la que los autores realizan la reducción de escala a una resolución inicial de $1^{\circ} (\sim 111 \text{ Km})$
para obtener datos georreferenciados a una resolución de 0.25$^{\circ}~(\sim 27 \text{.} 75 \text{ Km})$ en base a 4 diferentes fuentes de datos los cuales se encuentran a una alta resolución: 
\textit{GRACE} para el \textit{Terrestrial Water Sotrage}, \textit{GLDAS} para 4 diferentes variables hidrológicas (\textit{canopy water}, \textit{soil moisture}, 
\textit{snow water equivalent} y \textit{runoff}), \textit{GLEAM} para datos 
de vapor de agua emanado por plantas, el suelo, etc, conocida como \textit{evapotranspiration} y \textit{TRAMM} para datos de precipitación.

La premisa del procedimiento se basa en que las variables explicativas se encuentran a una resolución mucho mayor que los datos recopilados por GRACE,
por lo que en base al comportamiento de cada una de ellas es posible determinar las dinmámicas de la reserva de agua terrestre en puntos donde el satélite no 
logra captar anomalías más granulares.

El proceso de reducción de escala que se detalla en la figura \ref{metodologia}, en pocas palabras se puede describir
como:
\begin{itemize}

    \item Las 6 variables explicativas se remuestrean a $1^{\circ}$ (resolución 
    nativa de GRACE) y además a la resolución que se busca estimar o resolución objetivo (en este caso $0\text{.}25^{\circ}$), mediante 
    cierta interpolación, ya sea lineal, cúbica, etc.

    \item Se instancia el modelo \textit{Random Forest} regresivo. Con las variables explicativas a 1$^{\circ}$ se entrena y se valida
    con las observaciones de GRACE midiendo el error cuadrático medio y el error medio absoluto.
    
    \item Se generan las diferencias entre las predicciones del modelo y los valores reales para obtener
    datos llamados \textit{residuales} para posteriormente interpolarlos a la resolución objetivo.

    \item Con las 6 variables explicativas remuestradas a la resolución objetivo se predice el valor de \textit{TWS}
    y posteriormente se adiciona la data residual interpolada a la predicción, esto como una estrategia para introducir entropía.
    
    \item Finalmente, se substraen los valores de algunas variables explicativas para obtener el 
    \textit{Ground Water Storage}, o almacén de agua subterránea en cada punto de la grilla y validar estos valores con
    mediciones en terreno de pozos subterráneos u \textit{observations wells}.
    
\end{itemize}

%
%
%
%
\section{Recolección y preprocesamiento de datos}


La primera parte del trabajo consistió en la recolección de datos necesarios para replicar la metodología
propuesta en la sección 3.1, para esto se recurren a diferentes fuentes de data que nos otorguen observaciones igual
o de mejor calidad que las que se proponen.

En base a las variables explicativas presentes en \cite{11}, se proceden a investigar múltiples fuentes de datos, las cuales difieren de las utilizadas en el paper de referencia. 
Las bases de datos seleccionadas fueron \textbf{(CR)$^2$-met} para los datos de precipitación, \textbf{ERA5-Land} para las 5 variables explicativas restantes,
observaciones de pozos provienientes de la \textbf{DGA} y finalmente los datos de GRACE obtenidos de \textbf{OPeNDAP}.

El período de tiempo que considera el estudio comienza el mes de abril del año 2002 y culmina en febrero del año 2017 en instantes de tiempo en los cuales existan mediciones en cada una
de las variables.

En cuanto a los límites espaciales, el estudio es realizado en el parche que tiene como límites longitudinales desde 77.5°E hasta 66.5°E y latitudinales desde 16.5°S hasta
56.75°S, el cual corresponde a todo el territorio continental chileno.

    \subsection{GRACE}
    Los datos satelitales georrefereenciados de GRACE se encuentran empaquetados mensualmente durante todo el período de estudio
    para cualquier coordenada geográfica del globo. Por tanto el primer desafío ocurre al momento de extraer los datos de interés que son los  
    del territorio chileno, el cual resulta ser un proceso heurístico exhaustivo debido a la configuración de coordenadas (latitud, longitud) del sitio web,
    ya que la longitud se encontraba entre el intervalo $[0,180]$ cuando normalmente esta se encuentra entre los valores $[-90,90]$. Luego de múltiples intentos
    se consigue dar con las coordenadas que localizan al territorio como se muestra en la figura 3.2(a).
    
    \begin{figure}[H]
        \centering
              \subfigure[Datos de longitud y latitud para cada variable]{\includegraphics[height=3cm]{images/GRACE_webpage.png}\label{fig:vt:a}}
              \subfigure[Gráfico de GRACE]{\includegraphics[height=4.6cm]{images/GRACE_init.png }\label{fig:vt:b}}\goodgap
              \vskip -0.1in
        \caption[Rutina para extracción de datos GRACE]{Gráfico preliminar de datos de GRACE para las coordenadas de Chile continental en mes particular.}
        \label{logo}
    \end{figure}

    En la figura se aprecia que existen dos variables involucradas en el conjunto de datos: \textit{lwe\_thickness}, que es la abreviación del inglés
    \textit{level water equivalent thickness}, que se traduce como grosor de agua equivalente y \textit{uncertainty} la cual es la incertidumbre
    o el grado de error sobre la predicción del modelo. Las variables mencionadas poseen 3 características: tiempo, latitud y longitud.

    Entrando en el análisis cuantitativo, se puede graficar la distribución de datos de GRACE a baja resolución:
    \begin{figure}[h]
        \centering
              \includegraphics[scale=0.6]{images/distGRACEnoproc-1deg.png}
        \caption[Distribución de datos satelitales a baja resolución]{\footnotesize Distribución de datos satelitales GRACE sin procesar a baja resolución.}
    \end{figure}
    
    Claramente se percibe una distribución muy semejante a la normal, los valores de la media son 0.002110 y una desviación estándar de 0.05371, con valores extremales -0.375 como mínimo y 
    máximo de 0.484. Por ende, no es poible determinar el mismo criterio que en la figura \ref{logo1} ya que la escala de la zona de estudio es mucho menor, por esta razón, es mejor convenir las siguientes clasificaciones
    de acuerdo al valor de la anomalía:

    \begin{table}[H] 
        \caption[Rangos de valores de la anomalía TWS en Chile]{Representaciones del valor de la anomalía TWS del satélite GRACE para todo su dominio.}
        \newcolumntype{C}{>{\centering\arraybackslash}X}
        \begin{tabularx}{0.9\textwidth}{CCC}
        \toprule
        \textbf{Intervalo}	& \textbf{Categoría}	&\textbf{Método de cálculo}\\
            \midrule
            \textbf{$\left[-0\text{.}375,~-0\text{.}057\right[$}    & Más seco que el periodo anterior.       & $\left[v_{min},~ \mu - \sigma\right[$\\
            \textbf{$[-0\text{.}057,~0\text{.}0618]$}	            & Cercano a condiciones normales.         & $[\mu - \sigma,~ \mu +\sigma]$ \\
            \textbf{$\left]0\text{.}0618,~0\text{.}484\right] $}    & Más húmedo que el perídodo anterior.    & $\left]\mu +\sigma,~v_{max}\right]$\\
            \bottomrule
        \end{tabularx}
    \end{table}
    
    \subsection{ERA5-Land}
    ERA5-Land es un conjunto de datos que proporciona una visión coherente de la evolución de las variables terrestres durante varias décadas 
    con una resolución mejorada. ERA5-Land se ha producido reproduciendo el componente terrestre del reanálisis climático ECMWF ERA5. El reanálisis combina 
    datos de modelos con observaciones de todo el mundo en un conjunto de datos globalmente completo y consistente utilizando las leyes de la física. 
    Además, produce datos que se remontan varias décadas atrás en el tiempo, proporcionando una descripción precisa del clima del pasado \cite{ERA5-L}.

    El conjunto de datos son del tipo cuadriculado, con cuadrícula regular de latitud y longitud a una resolución de 0.1$^{\circ} (\sim 9$ Km), 
    para el proyecto en cuestión, de todo el territorio continental chileno.

    Las variables utilizadas como explicativas del modelo \textit{Random Forest} fueron la humedad del suelo (\textit{soil moisture}), nieve equivalente en agua (\textit{snow water equivalent}),
    evaporación por transpiración de la vegetación (\textit{evaporation from vegetation transpiration}), agua de dosel (\textit{canopy water}) y escorrentía (\textit{runoff}).
    
    \subsubsection{Evaporación por transpiración de la vegetación}
    Cantidad de evaporación de la transpiración de la vegetación. Esto tiene el mismo significado que la extracción de raíces, 
    es decir, la cantidad de agua extraída de las diferentes capas del suelo. 
    Esta variable se acumula desde el comienzo del tiempo de pronóstico hasta el final del paso de pronóstico y está medida en metros de agua equivalente \cite{ERA5-L}.

    \subsubsection{Escorrentía}
    Parte del agua de la lluvia, de la nieve que se derrite o de lo profundo del suelo, permanece almacenada en las capas subterráneas. De lo contrario, el agua se escurre, ya sea sobre la superficie (escorrentía superficial)
    o bajo tierra (escorrentía subterránea) y la suma de estos dos se denomina simplemente ``escorrentía''. Las unidad de medida de la escorrentía es la profundidad en metros. Esta es la profundidad que tendría el agua si se distribuyera uniformemente sobre el píxel de la rejilla.
    La escorrentía es una medida de la disponibilidad de agua en el suelo y puede, por ejemplo, utilizarse como indicador de sequía o inundación.

    \subsubsection{Nieve equivalente en agua}
    La profundidad que presenta la nieve en su área de cuadrícula se puede medir en metros de agua equivalente, por lo que coincide con la altura que tendría el agua líquida acumulada si la nieve se derritiera y se 
    repartiera uniformemente en cada pixel. El Sistema de Pronóstico Integrado ECMWF representa la nieve como una sola capa adicional sobre el nivel superior del suelo. La nieve puede cubrir todo o parte de la caja de rejilla.

    \subsubsection{Agua de dosel}
    El dosel (\textit{canopy}), consiste en el área formada por las copas de los árboles en un bosque. Por lo tanto, el agua de dosel consiste en el agua presente en cada una de las hojas de los árboles, dicha agua proviene de 
    la precipitación mayoritariamente, pero también se ve afectada por la humedad, la temperatura, entre otros factores.

    \subsubsection{Humedad del suelo}
    Esta variable explicativa consiste en la suma de 4 parámetros hidrológicos, éstos se definen como el volumen de agua presente en cada capa de suelo. La primera de ellas
    hace referencia al volumen de agua presente en la superficie (0 m) hasta los 0.07 m. La segunda capa parte en los 0.07 m y culmina en los 0.28 m. La tercera capa llega hasta el metro de profundidad partiendo desde los 0.28 m
    y la capa final, desde 1 metro hasta los 2.89 metros \cite{ERA5-L_doc}. Con estos parámetros se genera la variable \textit{soil moisture}.

    \subsection{Precipitación}
    El conjunto de datos CR2-MET contiene información meteorológica (precipitación, temperaturas medias y extremas) en un grilla rectangular de 0.05º latitud-longitud (aproximadamente 5 Km) para el territorio de Chile continental 
    durante el periodo de 1979-2016. La técnica utilizada para la construcción del producto de precipitación se basa en una regionalización estadística de datos del reanálisis atmosférico ERA-Interim (datos disponibles en grillas de $\sim$70 Km). 
    El método utiliza modelos estadísticos como funciones de transferencia para traducir precipitación, flujos de humedad y otras variables \cite{pr}.
    
    \subsection{Observaciones de pozos}
    Las observaciones de pozos de bombeo fueron otorgadas por 2 bases de datos diferentes de la DGA. Ambas bases de datos constaban de una serie de variables útiles para muchos estudios.
    Particularmente para el trabajo se utilizaron los códigos llamados \textbf{ESTID} \cite{gwl}, los cuales constan de códigos únicos que poseen una coordenada geográfica asociada localizada dentro
    del territorio chileno, la \textbf{fecha de la medición}, las cuales se encontraban distanciadas diariamente desde enero del 2000, hasta enero del año 2016 con valores faltantes y finalmente el valor de la \textbf{profundidad del pozo},
    la cual se encontraba en metros y representa la distancia desde la superficie terrestre hasta la superficie acuática de la cuenca, es decir, mientras mayor sea la prfundidad de un pozo, menos contenido de agua este presenta.

    Ambas bases de datos debieron corroborarse ya que estas poseían códigos ESTID comunes y aquellos que diferían en medición se optaba por descartarse debido a la inconsistencia presente.
    Finalmente el número de códigos que fueron sujeto a estudio fue de 1036. 

    \begin{figure}[H]
        \centering
              \includegraphics[scale=0.4]{images/Observacion_pozos.jpeg}
        \caption[Datos de Pozos DGA sin procesar]{\footnotesize Gráfica del valor promedio de cada código único para todo el intervalo de tiempo.}
        \label{pozosDGA}
    \end{figure}

    El objetivo de estos datos recae en validar el contenido de agua terrestre (TWS) a una alta resolución, es decir, corroborar las predicciones del modelo mediante el estudio de tendencias y estacionalidades a través de series de tiempo. 
    Como se muestra en la figura \ref{pozosDGA}, los datos provenientes de la DGA consideran solamente la zona norte y centro del país por lo que el estudio posterior de validación del
    \textit{downscalling} deberá estudidarse dentro de alguna de las zonas en las que existan mediciones en terreno.

    La primera intervención realizada a esta nueva base de datos unificada consistió en relacionar estos datos que se encuentran repartidos de manera no uniforme en el territorio. Es por esto que luego de aplicarle el remuestreo temporal detallado más adelante, el siguiente paso fue transformar los datos con un código ESTID asociado,
    a datos grillados. Donde las grillas serán la resolución original (1$^{\circ}$) y la grilla de alta resolución (0.25$^{\circ}$). Es claro que habrán pixeles de la grilla en donde 
    se encuentren más de una estación de medición y en otras donde no existan observaciones, es por esto que se debe adicionar nuevas variables, tales como \textbf{la densidad de pozos}, que se define como la cantidad de pozos presentes en cada pixel y además
    la \textbf{cantidad de observaciones}, es decir, la cantidad de datos presentes en cada punto de la grilla, esto para no perder aquellos píxeles que sean más representativos o que 
    contengan una alta riqueza en sus mediciones.

    \subsection{Remuestreo espacial}
    
    El remuestreo espacial fue una de las tareas más importantes a realizar durante el preprocesamiento. Este consiste en que a partir de los datos a cierta
    resolución nativa u original se buscan aumentar o disminuir dicha resolución. Para el rescalamiento espacial de los datos georreferenciados se optó por utilizar una herrmienta que sea capaz de convertir 
    cuadrículas de datos a la resolución deseada de forma rápida y que pueda operar con grillas predefinidas según las necesidades, por lo tanto CDO es el software indicado. 

    
    \begin{figure}[H]
        \centering
              \subfigure[Densidad a 0.25$^{\circ}$]{\includegraphics[scale=0.4]{images/wells_density025.png}}
              \subfigure[Densidad a 1$^{\circ}$]{\includegraphics[scale=0.4]{images/wells_density1.png }}\goodgap
              \vskip -0.1in
        \caption[Cantidad de pozos en cada grilla]{Gráfico de cantidad de pozos por píxel para grilla nativa y de alta resolución para el mes de enero del año 2005.}
        \label{pixeldepth}
    \end{figure}

    
    El rescalamiento espacial se realiza sobre datos con una alta resolución espacial, en este caso las variables explicativas de ERA5-L y precipitación, luego, mediante interpolación bilinear a través de una rutina en \textit{bash} de CDO se consigue
    la resolución objetivo (0.25$^\circ$) de\textit{Downscaling} y la resolución original de GRACE (1$^\circ$). Es claro que hay muchas maneras de formar las grillas de datos los cuales se encuentren a una resolución de 
    1$^\circ$ y que éstas no necesariamente tienen que ser las mismas entre ellas, es por esto que se fija la grilla a baja resolución y alta resolución antes de realizar el remuestreo
    para que todos los datos se encuentren remuestrados por una misma grilla. Lo más sencillo es extraer los datos de GRACE y de ellos obtener las coordenadas georreferenciadas a resolución de 1$^\circ$ y así se disminuye la cantidad de
    variables a remuestrar.

    Aunque CDO entregue resultados mucho más rapido que Pyhon, la mayor desventaja recae en el hecho que el software es ajeno al código en el cual se han realizado todos y cada uno de los análisis anteriores, por tanto se deberían realizar otros procedimientos para poder integrar las operaciones a los archivos resultantes.
    A partir de esto, se opta por realizar interpolación bilinear de forma interna en Python y, para corroborar los resultados, se recurre a CDO que al ser un software especializado nos entrega seguridad a la hora de realizar el remuestreo
    espacial. Se escoge la variable hidrológica \textit{runoff} proveniente del conjunto de datos ERA5-L y se procede a realizar las interpolaciones para el mes de marzo del año 2009 y se obtuvieron los siguientes resultados:
    
    \begin{table}[H] 
        \caption[Comparación de interpolaciones Python/CDO]{Comparación de interpolación bilinear realizada en software especializado (CDO) versus Python desde la resolución original de 0.1° a 0.25°.}
        \newcolumntype{C}{>{\centering\arraybackslash}X}
        \begin{tabularx}{0.9\textwidth }{CCCC}
        \toprule
         \textbf{Datos}& \textbf{Intervalo}	&\textbf{Media ($\mu$)}  &\textbf{Desviación estándar ($\sigma$)}\\
            \midrule
            \textbf{Originales 0.1°}	&	[0, 0.01130]       &  0.0002417  & 0.000788\\
            \textbf{CDO 0.25°}	        &   [0, 0.01058]       &  0.0002242  & 0.000755\\
            \textbf{Python 0.25°}       &   [0.00024, 0.00083] &  0.0002430  & 0.000831\\
            \bottomrule
        \end{tabularx}
    \end{table}

    Si bien las cotas de los datos para Python se ven mucho más restringidas, este análisis no nos entrega información de qué tan cercanas son cada una de las interpolaciones entre sí. Claramente CDO será el punto de comparación debido a que
    la función para la cual el software fue diseñado, es operar en grillas de datos, el objetivo ahora es calcular qué tan lejos se encuentra la interpolación de Python con la de CDO.
    Así, se proceden a calcular los errores ya conocidos, \textit{MAE} y \textit{RMSE}. Obteniendo un valor de error medio absoluto de $1.93317\times 10^{-6}$ y un error cuadrático medio de $5.262181\times 10^{-10}$. Por lo tanto, 
    la interpolación realizada en Python resulta ser válida en base a la interpolación que se realiza utilizando CDO.


    \subsection{Remuestreo temporal}
    Debido a la naturaleza de los datos de precipitación y de mediciones de pozos de bombeo, fue necesario realizar un preprocesamiento 
    adicional al de ERA5-L ya que además del remuestreo espacial a 1$^\circ$ y 0.25$^\circ$ se debió realizar un remuestreo temporal
    mensual para adicionar la variable al modelo.

    Junto con la alta riqueza espacio-temporal de la base de los productos grillados de precipitación surge un problema no menor: la capacidad de procesamiento para los datos aumenta
    exponencialmente, por este motivo se busca realizar el procesamiento temporal mediante la librería \texttt{xarray} de Python y no en \texttt{pandas} como se había hecho con
    ERA5, esto para dar paso al procesamiento espacial el cual se debió realizar mediante la librería de \texttt{geopandas} debido a la naturaleza misma de los datos
    los cuales se encuentran georreferenciados.

    Debido a que la naturaleza de las observaciones de pozos, el problema de capacidad de cómputo desaparece y no resulta problema remuestrear mensualmente el valor de la profundidad a través de 
    un promedio simple para cada código de medición.
%   
%
%
%
%
%
%
\section{Entrenamiento y Testeo}
    Para entrenar el modelo Random Forest, en primera instancia es necesario dividir el conjunto de datos en entrenamiento
    y testeo, para luego verificar la predicción en base a observaciones que no han sido utilizadas en el entrenamiento. Debido a que las variables explicativas son hidrológicas y 
    el territorio nacional presenta diversos climas en toda su extensión, surge la idea de estratificar los datos de entrenamiento y testeo mediante las macrozonas del país para conseguir predicciones coherentes 
    y que la división de conjuntos no se encuentre sesgada de ninguna manera.
    \begin{figure}[H]
        \centering
              \includegraphics[width=\textwidth]{images/division_zonas.png}
              \vskip -0.1in
        \caption[División de datos por macrozonas de Chile]{\footnotesize División de datos en base a las zonas de Chile.}
        \label{division}
    \end{figure}

    \begin{figure}[H]
        \centering
            \includegraphics[width=\textwidth]{images/traintest.png}
              \vskip -0.1in
        \caption[Entrenamiento y testeo mediante macrozonas]{\footnotesize División en entrenamiento y testeo para cada macrozona de Chile.}
        \label{traintest}
    \end{figure}

    En la figura \ref{division}, \texttt{DATA} constituye todas las variables explicativas para el modelo con mediciones remuestradas en la grilla de GRACE
    a 1$^{\circ}$, el valor de la variable respuesta (\texttt{lwe\_thickness}), además de la respectiva fecha de medición (\texttt{time}) y 
    coordenadas geográficas (\texttt{lon} y \texttt{lat})
%
%
%
%
\section{Construcción del modelo Random Forest}
Según el sitio web de sklearn \cite{rfskl}, un bosque aleatorio es un metaestimador 
que ajusta una serie de árboles de decisión en varias submuestras 
del conjunto de datos y utiliza el promedio para mejorar la precisión predictiva y controlar el sobreajuste. 
El tamaño de la submuestra se controla con el parámetro \texttt{max\_samples}, si el parámetro \texttt{bootstrap=True}, el cual se encuentra predeterminado, 
de lo contrario, se usa todo el conjunto de datos para construir cada árbol.

Para la construcción del modelo se crearon 2 predictores, uno de ellos minimiza el error cuadrático medio (fig. \ref{logo2}) y el otro minimiza el error medio absoluto (fig. \ref{logo3}). En cuanto a la construcción misma, ambos modelos comparten los mismos hiperparámetros, 
éstos son:
\begin{itemize}
    \item \texttt{n\_estimators}: Es la cantidad de árboles presentes en el \textit{ensemble}. 
    \item \texttt{max\_depth}: Se define como la profundidad del árbol, es decir, la cantidad máxima de particiones de cada estimador.
    \item \texttt{max\_features}: Es la cantidad de características que se consideran para encontrar la mejor división del árbol, o en otras palabras, el porcentaje de variables explicativas utilizadas para realizar la mejor división en cada árbol. \texttt{None} 
    se refiere a que el algoritmo seleccione la profundidad del árbol de acuerdo a la minimización de su función objetivo.
    \item \texttt{oob\_score}: La abreviación para \textit{out of bag score} definida como el porcentaje de observaciones las cuales no serán utilizadas para entrenar el modelo a la hora de hacer la muestra de \textit{boostrap} y por tanto solo serán utilizadas para testear.
    \texttt{False} significa que el modelo utilizará todo el conjunto para realizar el \textit{bagging}.
    \item \texttt{n\_jobs}: Hace alusión a la cantidad de procesadores virtuales que se utilizarán para entrenar el modelo gracias a que el \textit{bagging} es fácilmente paralelizable y en Python se encuentra implementada esta función, en este caso se le asigna el valor de  
    $-1$ para que se utilicen todos los procesadores disponibles y así disminuir el tiempo de cómputo.
    \item \texttt{random\_state}: Define la semilla con la cual se generarán las muestras aleatorias, por lo que si se le otorga un valor se puede obtener un modelo replicable.
\end{itemize}

\begin{figure}[H]
    \centering
          \includegraphics[width=\textwidth]{images/RF_regressor.png}
          \vskip -0.1in
    \caption[Construcción y entrenamiento del modelo Random Forest con criterio de \textit{mse}]{\footnotesize Modelo Random Forest construído en Python con minimización del error cuadrático medio.}
    \label{logo2}
\end{figure}

\begin{figure}[H]
    \centering
          \includegraphics[width=\textwidth]{images/RF_regressor_mae.png}
          \vskip -0.1in
    \caption[Construcción y entrenamiento del modelo Random Forest con criterio de \textit{mae}]{\footnotesize Modelo Random Forest construído en Python con minimización del error medio absoluto.}
    \label{logo3}
\end{figure}

    \subsection{Interpolación de Residuos}

    Posterior a instanciar cada uno los modelos y seleccionar el que mejor métricas de validación entregue en la resolución nativa, corresponde calcular la diferencia entre los valores reales y los valores predichos, es decir, los datos residuales.
    Luego de obtener dichos datos es necesario remuestrearlos a la resolución objetivo, esto mediante interpolación bilinear.
 
    Mediante dos funciones creadas en Python y la clase \texttt{interpolate.interp2d} de la librería \texttt{scipy} es posible interpolar los datos a partir de sus 4 vecinos más cercanos resolviendo el sistema de ecuaciones
    de la ecuación \ref{interp2d}, estas funciones son:
    \begin{itemize}
        \item {\texttt{nearest\_coor\_patch}: Es una función que particiona la grilla de longitud (\texttt{df\_lon}) y de latitud (\texttt{df\_lat}) del \textit{dataframe} y retorna los 4 vecinos más cercanos en base a la distancia euclídea
        de las coordenadas de longitud y latitud entregadas.}
        \item {\texttt{bilinear\_interpolation}: Función que interpola el parche de coordenadas entregado (\texttt{patch}) de la variable \texttt{var\_name} en la longitud (\texttt{new\_lon}) y latitud (\texttt{new\_lat}) dadas. En el caso de haber un error cuantitativo de interpolación,
         la función retorna el objeto nulo de la librería \texttt{numpy}.}
    \end{itemize}
    \begin{figure}[H]
        \centering
              \includegraphics[width=\textwidth]{images/res_interp.png}
              \vskip -0.1in
        \caption[Funciones utilizadas para la interpolación de residuos]{\footnotesize Funciones utilizadas para interpolar bilinearmente la data residual de 1$^{\circ}$ a 0.25$^{\circ}$.}
        \label{logo4}
    \end{figure}

    \begin{figure}[H]
        \centering
              \includegraphics[width=\textwidth]{images/res_interp_rutine.png}
              \vskip -0.1in
        \caption[Rutina de interpolación de residuos]{\footnotesize Rutina para interpolar bilinearmente la data residual de 1$^{\circ}$ a 0.25$^{\circ}$.}
        \label{logo5}
    \end{figure}

    Posteriormente, se toman las grillas mensuales y para cada coordenada en la grilla a baja resolución se procede a interpolar en base a la funciones antes mencionadas. La rutina de interpolación de residuos se detalla en la figura \ref{logo5} para almacenar los valores residuales en la variable \texttt{data\_res}.
    
    Finalmente, los datos residuales interpolados son añadidos a la predicción a alta resolución como una forma de agregar entropía al modelo y así otorgarle un grado de intertidumbre al valor final.



   

\chapter{Resultados y Discusión}
\label{C4}
Los resultados debieron ser analizados de dos formas: Una de ellas consiste en la medición del error de cada modelo instanciado en el conjunto de test, estos errores son la 
raíz del error cuadrático medio \ref{eqn:mse}, en inglés \textit{Root Mean Squared Error (RMSE)} y el error medio absoluto (\textit{MAE}) \ref{eqn:mae}, ésto significa una validación en la resolución nativa de 
GRACE, es decir 1$^\circ$. La segunda manera es la validación en la alta resolución, es decir, posterior a realizar la predicción a 0.25$^\circ$; Está claro que no se posee
un conjunto para validar los datos en este escenario debido a que sería un \textit{Downscalling} ya hecho. Por este motivo se deben optar por otras maneras de corroborar la información
para que ésta sea fiable, aquí es donde las series de tiempo se ajustan a lo que buscamos validar.

\section{Métricas de validación en el conjunto de prueba }
Como se ha mencionado, la primera validación ocurre en la resolución nativa de GRACE. Ahora, en primera instancia se evalúa el rendimiento general de cada modelo, es decir, 
se calcula el error medio absoluto y \textit{RMSE} para el conjunto entero de testeo. Con este procedimiento se obtuvieron los siguientes errores:

\begin{itemize}
    \item Para el modelo que \textbf{minimiza el error cuadrático medio}, la métrica \textit{MAE} es de aproximandamente 0,0056904 y un \textit{RMSE} con valor de 6,23$ \times 10^{-5}$.
    \item Para el modelo que \textbf{minimiza el error medio absoluto}, la métrica \textit{MAE} es de 0,00615 aproximadamente y el \textit{RMSE} se obtiene un valor de 7,25$ \times 10^{-5}$.
\end{itemize}

Para ambos modelos, las variables explicativas más importantes fueron \texttt{time}, \texttt{lat}, \texttt{lon}, humedad de suelo (\texttt{sm}) y la evaporación por transpiración de la vegetación (\texttt{e}), en ese orden. Claramente el porcentaje de importancia
varía pero el orden se mantiene. Por lo tanto, se puede afirmar que los modelos se encuentran afectos al rendimiento mayoritariamente en base al criterio de minimización escogido.

La siguiente validación en base a la resolución nativa de GRACE es con respecto a las macrozonas del país, es decir, aquellos datos que fueron estratificados y luego separados en entrenamiento y testeo (figura \ref{traintest}).
En cada macrozona se calculan las métricas de error correspondiente en el conjunto test. Los resultados se muestran en la siguiente tabla:

    \begin{table}[H] 
        \caption[Métricas de validación para cada macrozona]{Métricas de validación para cada macrozona de Chile.}
        \newcolumntype{C}{>{\centering\arraybackslash}X}
        \begin{tabularx}{0.9\textwidth}{CCCCC}
        \toprule
        \textbf{zona}	& \textbf{MAE \texttt{mse\_model}}	& \textbf{MAE \texttt{mae\_model}} & \textbf{RMSE \texttt{mse\_model}} & \textbf{RMSE \texttt{mae\_model}}\\
            \midrule
            \textbf{N. grande}		& 0.004583 & 0.005005 & 0.000043 & 0.000050\\
            \textbf{N. chico}		    & 0.005522 & 0.006020 & 0.000057 & 0.000068\\
            \textbf{Centro}             & 0.006888 & 0.007355 & 0.000084 & 0.000096\\
            \textbf{Sur}                 & 0.005722 & 0.005968 & 0.000061 & 0.000066\\
            \textbf{Austral}            & 0.005963 & 0.006510 & 0.000068 & 0.000082\\
            \bottomrule
        \end{tabularx}
    \end{table}

Para comparar cada modelo en base a las métricas de evaluación, conviene realizar un gráfico, en este caso se realiza uno por cada métrica debido a que el orden de los errores
difieren. 

\begin{figure}[H]
    \centering
          \includegraphics[scale=0.78]{images/RMSE_RFmodels.png}
          \vskip -0.1in
    \caption[Error cuadrático medio en cada macrozona del país]{\footnotesize Medición del error cuadrático medio en cada macrozona de Chile.}
    \label{mserf}
\end{figure}

\begin{figure}[H]
    \centering
          \includegraphics[scale=0.78]{images/MAE_RFmodels.png}
          \vskip -0.1in
    \caption[Error medio absoluto en cada macrozona del país]{\footnotesize Medición del error medio absoluto en cada macrozona de Chile.}
    \label{maerf}
\end{figure}

Es evidente que el modelo el cual minimiza el error cuadrático medio, además de reducir la varianza en la predicción, mejora la predicción misma otorgando un menor error tanto
general como en cada macrozona del país, mostrando la superioridad de la minimización del error cuadrático medio por sobre el error medio absoluto. Es por esta razón, que el modelo escogido para 
predecir valores de \textit{lwe\_thickness} provenientes de GRACE a una mayor resolución será el modelo que mejores métricas de evaluación presente.
%
%
%
%
\section{Resultados Finales}
Es necesario señalar que los resultados que se presentan a continuación son válidos para el territorio continental chileno, debido a la naturaleza de las bases de datos, específicamente de ERA5-Land y precipitación, es 
por este motivo que las observaciones presentes fuera del territorio continental chileno no deben utilizarse para estudios posteriores.

Como se menciona en la metodología del paper base \cite{11}, luego de obtener el TWS final a este se le deben substraer aquellas variables hidrológicas superficiales para así
obtener el contenido de aguas subterráneas, no obstante, no se puede asegurar que realizando la substracción de variables hidrológicas superficiales se consiga el valor del almacenamiento de aguas subterráneas, o 
\textit{Ground Water Storage}, esto debido a muchos otros factores que se ignoran a la hora de realizar el procedimiento, es por esta razón que se decide omitir este paso y proceder a la validación con mediciones \textit{in situ} o mediciones en terreno con datos de pozos de bombeo que 
se detalla más adelante.

Luego de la adición de los residuos interpolados, son 156 las grillas de datos obtenidas desde abril del año 2002 hasta junio del año 2017. A continuación se muestran los datos con reducción de escala para
uno de los primeros períodos de tiempo, específicamente septiembre del año 2002, a mitad del período y para finales del mismo, es decir, abril del año 2017.

\begin{figure}[H]
    \centering
          \subfigure[GRACE a 1$^{\circ}$]{\includegraphics[scale=0.53]{images/Before_Downscalling_v0-09-02.jpeg}}
          \subfigure[TWS a 0.25$^{\circ}$ ]{\includegraphics[scale=0.53]{images/Downscalling_v0-09-02.jpeg}}\goodgap
          \vskip -0.1in
    \caption[\textit{Downscalling} final para el mes 09/2002]{Gráfico del aumento de resolución de GRACE a resolución de 0.25$^{\circ}$ para septiembre del año 2002.}
    \label{dsf02}
\end{figure}


\begin{figure}[H]
    \centering
          \subfigure[GRACE a 1$^{\circ}$]{\includegraphics[scale=0.53]{images/Before_Downscalling_v0-06-14.jpeg}}
          \subfigure[TWS a 0.25$^{\circ}$ ]{\includegraphics[scale=0.53]{images/Downscalling_v0-06-14.jpeg}}\goodgap
          \vskip -0.1in
    \caption[\textit{Downscalling} final para el mes 06/2014]{Gráfico del aumento de resolución de GRACE a resolución de 0.25$^{\circ}$ para junio del año 2014.}
    \label{dsf14}
\end{figure}

A simple vista, se logra apreciar que los mapas de calor son coincidentes en cuanto a los valores de las anomalías se refiere, es decir, el modelo es capaz de capturar correctamente
las zonas en las cuales existe un déficit o superhábit hidrológico en base a las mediciones originales y las variables hidrológicas. Aunque esto parezca concluyente, aún no es suficiente para dterminar si el \textit{Downscalling}
ha sido exitoso, es por esta razón que aún queda la validación de datos de GRACE a alta resolución pendiente.

\begin{figure}[H]
    \centering
          \subfigure[GRACE a 1$^{\circ}$]{\includegraphics[scale=0.53]{images/GRACE_mapv1-04-17.jpeg}}
          \subfigure[TWS a 0.25$^{\circ}$ ]{\includegraphics[scale=0.53]{images/Downscalling_mapv1-04-17.jpeg}}\goodgap
          \vskip -0.1in
    \caption[\textit{Downscalling} final para el mes 06/2014]{Gráfico del aumento de resolución de GRACE a resolución de 0.25$^{\circ}$ para abril del año 2017.}
    \label{dsf17}
\end{figure}



Finalmente, podemos comparar las distribuciones de la resolución nativa de GRACE con el \textit{Downscalling} posterior a la adición de residuos,
utilizando la librería \texttt{seaborn} es poisble obtener dichas densidades, en base a ello se obtienen las gráficas presentes en la figura \ref{distcomp}.
\begin{figure}[H]
    \centering
          \subfigure[Distribución TWS a 1$^{\circ}$]{\includegraphics[scale=0.6]{images/distGRACE1deg.png}}
          \subfigure[Distribución TWS a 0.25$^{\circ}$ ]{\includegraphics[scale=0.6]{images/distGRACE025deg.png}}\goodgap
          \vskip -0.1in
    \caption[Comparativa de distribuciones de GRACE]{Gráfico de las distribuciones de los datos originales de GRACE con los valores predichos a alta resolución con el modelo Random Forest.}
    \label{distcomp}
\end{figure}

Es evidente que el valor de las densiades (eje $y$) difieren en demasía debido al mayor volumen de datos presente en las grillas a alta resolución, de todas maneras, ambas distribuciones
se asemejan y mantienen un coportamiento normal. Ahora, si calculamos ciertos estadísticos útiles, tales como media, desviación estándar y el rango donde se mueve las anomalías a alta y baja resolución podremos tener una idea más clara 
en cuanto a la similitud de distribuciones se trata.

\begin{table}[H] 
    \caption[Comparación de distribuciones de TWS a baja y alta resolución]{Comparación de TWS a baja y alta resolución en distribución, media, desviación estándar y soporte.}
    \newcolumntype{C}{>{\centering\arraybackslash}X}
    \begin{tabularx}{0.9\textwidth}{CCCCC}
    \toprule
    \textbf{resolución}	& \textbf{mínimo}	& \textbf{máximo} & \textbf{media ($\mu$)} & \textbf{desviación estándar ($\sigma$)}\\
        \midrule
        \textbf{baja (1$^{\circ}$)}		& -0.282633 & 0.229261  & 0.000818 & 0.058005\\
        \textbf{alta (0.25$^{\circ}$)}   & -0.284269 & 0.233349	& 0.001885 & 0.057145\\
        \bottomrule
    \end{tabularx}
    \label{distcomp}
\end{table}

De la tabla \ref{distcomp}, se puede deducir que el \textit{Downscalling} ha sido realizado de manera correcta, manteniendo la distribución de los datos originales, aumentando ligeramente el rango en el que éstos se encuentran,
aunque con la media desplazada más lejana del origen y con una pequeña disminución en la variabilidad.


%
%
%
%


\section{Estudio y validación con series de tiempo}

\subsection{Zona de estudio}

Debido a la extensa hidrografía presente en el territorio chileno y de los limitados datos de observaciones de pozos, la validación mediante mediciones \textit{in situ} se centrará
en la región de Coquimbo en el periodo donde se puedan comparar las tendencias de GRACE con las observaciones de pozos, es decir, desde abril del año 2002 hasta enero del año 2016. 

En primera instancia, se realiza un estudio general de datos promediados regionales, 
posteriormente se escoge un píxel de baja resolución el cual contiene las observaciones de pozos remuestradas para finalmente estudiar aquellos píxeles a alta resolución que contengan zonas agrícolas \cite{LC}, las cuales necesitan agua para 
poder realizar el regadío. Este es un factor que se tendrá presente a la hora de realizar el estudio de tendencias para GRACE escalado y las observaciones de pozos.

\begin{figure}[H]
    \centering
          \includegraphics[scale=0.6]{images/Wells_with_landcover_coq.jpeg}
          \vskip -0.1in
    \caption[Zona de validación con series de tiempo]{\footnotesize Zona donde se valida el \textit{Downscalling} a través de series de tiempo con las observaciones de pozos medida en metros de profundidad y además las zonas de uso agrícola resaltadas con tonalidad rosa.}
    \label{szcoq}
\end{figure}

\subsection{Validación con series de tiempo}
El estudio tendencial de las series de tiempo comienza de forma general, es decir, conociendo y estudiando las anomalías porcentuales (ecuación \ref{pct_anomalies}) de las series temporales regionales para todo el período de tiempo, para este estudio se consideran 3 variables: En primer lugar las \textbf{anomalías de TWS} de alta y baja resolución promedio, en segundo lugar la 
\textbf{precipitación} y finalmente las \textbf{observaciones de pozos} individuales de la región graficadas en color gris y su promedio mensual.

\begin{figure}[H]
    \centering
          \includegraphics[width=\linewidth]{images/Anomalies_coq_reg.png}
          \vskip -0.1in
    \caption[Series de tiempo para la región de Coquimbo]{\footnotesize Anomalías porcentuales del promedio mensual de la región de Coquimbo.}
    \label{tscoq}
\end{figure}
En la serie temporal se puede apreciar que las tendencias de GRACE original y GRACE escalado son muy similares aunque no del todo; Se puede observar que los cambios porcentuales difieren, sobre todo desde el año 2010 hasta principios de año 2013. Para la precipitación,
es esperable encontrarse con la estacionalidad que presenta para los periodos invernales en el cono sur del globo, además, comienza con una tendencia en descenso y finaliza con un aumento notorio desde principios del 2013, al igual que las anomalías de TWS. Para los datos promediados de pozos, 
se puede observar que la tendencia señala un vaciado constante de los depósitos de aguas subterráneas con un aumento considerable al mediados del año 2015, tendencia la cual se puede explicar con la disminución tendencial del contenido de 
agua terrestre para la región, aún así, la región en general presenta un déficit anual en las aguas subterráneas que es consistente con la localización de las observaciones \textit{in situ} dentro o muy cercanas a las zonas agrícolas. 

En base a la figura \ref{szcoq} es posible identificar aquellas zonas en las cuales existen actividades que requieren un constante y a la vez considerable suministro de agua. Es posible observar las zonas donde existe una alta cantidad de pozos, zonas agrícolas y comparar
tendencialmente las series de tiempo; Para cumplir ese objetivo, se debe identificar el píxel a baja resolución y los píxeles a alta resolución que se encuentren contenidos.

\begin{figure}[H]
    \centering
          \includegraphics[scale=0.6]{images/Pixel1_and_pixels025_to_validate.png}
          \vskip -0.1in
    \caption[Píxeles a 0.25° y 1° que contienen zonas agrícolas y alta densidad de pozos]{\footnotesize Píxeles a 0.25° y 1° que contienen zonas agrícolas y alta densidad de pozos para validar a través de series de tiempo.}
    \label{pixelscoq}
\end{figure}
En la figura \ref{pixelscoq}, se muestra el píxel escogido a baja resolución que posee como centroide las coordenadas (-71.5,-30.5) el cual contiene 27 códicos únicos de pozos, con observaciones en su mayoría dentro de zonas agrícolas. Además, se destacan dos píxeles a alta resolución con centroide en las coordenadas 
(-71.25,-30.75) y (-71.25,-30.5) en color azulado que contienen una densidad de pozos de 9 y 4 respectivamente, por tanto, es posible determinar el impacto
que poseen las zonas agrícolas sobre las mediciones de pozos y comparar el comportamiento tendencial de las anomalías de TWS con el objetivo de dilusidar con más detalle las dinámicas granulares de las variables.


\begin{figure}[H]
    \centering
          \includegraphics[width=\linewidth]{images/Anomalies_pixel-7125S-3075W_0.png}
          \vskip -0.1in
    \caption[Series de tiempo para pixel particular a 1° con pixel a 0.25° contenido]{\footnotesize Anomalías porcentuales para pixel particular a 1° (-71.25, -30.75) con pixel a 0.25° (-71.25,-30.75) contenido.}
    \label{tspixel0}
\end{figure}

\begin{figure}[H]
    \centering
          \includegraphics[width=\linewidth]{images/Anomalies_pixel-7125S-3075W_1.png}
          \vskip -0.1in
    \caption[Series de tiempo para pixel particular a 1° con pixel a 0.25° contenido]{\footnotesize Anomalías porcentuales para pixel particular a 1° (-71.25, -30.75) con pixel a 0.25° (-71.25, -30.5) contenido.}
    \label{tspixel1}
\end{figure}

En cuanto al primer pixel a alta resolución (fig. \ref{tspixel0}), a pesar de que la tendencia de precipitación desde el año 2013 presenta un aumento, no se logra apreciar el impacto que esta tiene sobre las observaciones de GRACE
ya que para el período en cuestión, las anomalías dan a entender la disminución en el contenido de agua terrestre, información que es corroborable con las observaciones de pozos, donde en el píxel a baja resolución se tiene una 
tendencia de vaciado del pozo inclusive cuando en el periodo inicial, es decir, desde el 2002 hasta el 2004 tanto TWS como la profundidad de pozos muestran un aumento en la cantidad de agua. Un comportamiento más inusual sucede con el pozo a alta resolución,
el cual presenta un nivel tendencial constante hasta el comienzo del período de sequía de GRACE (2012) donde el nivel del pozo tiende a disminuir rápidamente hasta el año 2015, para finalmente aumentar en un $120\%$ aproximadamente en tan solo un año. 

Al analizar la figura \ref{tspixel1} la gran diferencia recae en las observaciones de pozos, donde estas presentan una tendencia negativa desde el año 2011, lo que se traduce en una disminución en el nivel de agua, tendencia que se puede
explicar por la sequía presente desde el año 2002 que experimenta GRACE.

Finalmente, en ambas figuras se puede apreciar que las tendencias entre GRACE a 1° se mantienen en cualquiera de los pixel a 0.25°, por lo que el \textit{Downscalling} además de asemejarse en distribución, también lo hace 
en la media temporal de las observaciones.


\subsection{Correlación Temporal}
En cuanto al territorio continental chileno se trata, el coeficiente de correlación definido en la ecuación \ref{eqn:cort} para los datos promediados mensuales entre la resolución original (GRACE a 1°) y la resolución objetivo (\textit{Downscalling} a 0.25°) fue de
\textbf{0.6544517} valor cercano a 1, lo cual indica que las tendencias poseen comportamientos similares y cuando una de ellas aumenta, la otra también y viceversa.

Finalmente, para el analisis tendencial general de la cuarta región de Coquimbo, se calcula el índice de correlación de Pearson 
para el promedio de Grace original a baja resolución y los datos con aumento de resolución, ambos normalizados mediante la normalización min-max. El valor del índice 
fue de \textbf{0.65441778}, un poco más bajo que el coeficiente naccional pero es más que aceptable para mencionar que los comportamientos tendenciales del \textit{Downscalling} se mantienen
en base al satélite GRACE original.

%
%
%
%

\section{Inerpolación bilinear y estudio de variabilidad}
El último componente del estudio del \textit{Downscalling} mediante un método de \textit{ensemble} basado en árboles de decisión es la comparación de una interpolación bilinear de los datos de GRACE a la resolución objetivo
con el aumento de resolución realizada mediante el método de Random Forest. Para realizar la interpolación se opta por utilizar el software CDO y como un estudio \textit{a priori} del \textit{Downscalling} se miden los tiempos de remuestreo
espacial de CDO y la predicción realizada por el modelo predictivo en la máquina virtual \texttt{ e2-custom-6-2176}, los tiempos de ejecución fueron de 0.37 y de 4.96 segundos respectivamente.

\begin{figure}[H]
    \centering
          \subfigure[Interpolación bilinear]{\includegraphics[scale=0.53]{images/Downscalling_cdo-09-02.jpeg}}
          \subfigure[TWS a 0.25$^{\circ}$ ]{\includegraphics[scale=0.53]{images/Downscalling_v0-09-02.jpeg}}\goodgap
          \vskip -0.1in
    \caption[Comparación de interpolación bilinear y \textit{Random Forest}]{Gráfico de interpolación bilinear con CDO y la predicción de TWS del modelo \textit{Random Forest} para septiembre del año 2002.}
    \label{interp02}
\end{figure}

Posterior a realizar la interpolación bilinear, los productos grillados obtenidos mediante CDO y el modelo predictivo son promediados por cada píxel en el territorio para así obtener una única grilla que representa los comportamientos temporales promedio. Para la interpolación bilinear se 
obtuvo una varianza de 1,7963$\times 10^{-4}$ y para la regresión 1,3850$\times 10^{-4}$. 

Posterior al cálculo de las varianzas,
se calculan las diferencias absolutas por píxel entre ambas grillas y posteriormente seleccionar $50\%$ de los píxeles que poseen mayor diferencia. Del total de píxeles de la grilla, un $6.7\%$ de los píxeles se encuentran con una mayor diferencia entre el algoritmo predictivo y la interpolación bilinear.
Es posible aislar la zona donde se encuentra el mayor error y poder realizar una comparativa de las dinámicas presentes.

\begin{figure}[H]
    \centering
          \includegraphics[width=0.8\textwidth]{images/area_with_high_differences_cdorf.png}
          \vskip -0.1in
    \caption[Área donde la interpolación bilinear y el modelo predictivo difieren considerablemente]{Píxeles donde existe una alta diferencia (destacados en rojo) entre la interpolación bilinear con CDO y el modelo predictivo \textit{Random Forest}}
    \label{high_diff}
\end{figure}
En la zona delimitada en la figura \ref{high_diff}, se aprecia que en los píxeles donde el modelo difiere de la interpolación es mayoritariamente en donde existe océano. Para obtener un marco de referencia, se escoge el parche en el cual no existen diferencias 
considerables y así estudiar las varianzas de cada variable explicativa, la interpolación bilinear y la prediccón del modelo  \textit{Random Forest}.

\begin{table}[H] 
    \caption[Varianzas en diferentes parches de CDO y RF]{Varianzas en parches con altas y bajas diferencias promedio entre interpolación bilinear con CDO y regresión mediante RF}
    \newcolumntype{C}{>{\centering\arraybackslash}X}
    \begin{tabularx}{0.9\textwidth}{CCC}
    \toprule
    \textbf{Parche}	& \textbf{Varianza CDO}	& \textbf{Varianza \textit{Random Forest}} \\
        \midrule
        \textbf{Altas diferencias}		& 1,1664$\times 10^{-6}$ &3,03805$\times 10^{-6}$  \\
        \textbf{Bajas diferencias}		 & 9,665881$\times 10^{-6}$ & 9,351364 $\times 10^{-6}$\\
        \bottomrule
    \end{tabularx}
\end{table}
\newpage

Es claro que en el parche en cuestión, las anomalías de TWS a baja resolución que entrega el modelo \textit{Random Forest} entrega mucha más variabilidad en la predicción que la interpolación bilinear, pero fuera de ésta zona
la variabilidad del modelo predictivo como de la interpolación bilinear se mantienen cercanas.

\begin{figure}[H]
    \centering
          \subfigure[Interpolación bilinear]{\includegraphics[scale=0.75]{images/rfvbil_2002.png}}
          \subfigure[TWS a 0.25$^{\circ}$ ]{\includegraphics[scale=0.75]{images/bilvrf_2002.png}}
          \vskip -0.1in
    \caption[Área donde la interpolación bilinear y el modelo predictivo difieren en el periodo 09/2002]{Píxeles donde existe una alta diferencia entre la interpolación bilinear con CDO y el modelo predictivo \textit{Random Forest} para septiembre del año 2002.}
    \label{bilrf2002}
\end{figure}

Al momento de observar un mes particular del parche de píxeles de la figura \ref{high_diff} es posible identificar las zonas donde la regresión difiere de la interpolación, por ejemplo para el mes de septiembre del año 2002 (fig. \ref{bilrf2002})
los píxeles oceánicos poseen un valor más bajo en la anomalía de TWS indicando una mayor estabilidad del contenido de agua en esas zonas. El mismo comportamiento también es capturado para el mes de junio del año 2014 (fig. \ref{bilrf2014}).

\begin{figure}[H]
    \centering
          \subfigure[Interpolación bilinear]{\includegraphics[scale=0.75]{images/rfvbil_2014.png}}
          \subfigure[TWS a 0.25$^{\circ}$ ]{\includegraphics[scale=0.75]{images/bilvrf_2014.png}}
          \vskip -0.1in
    \caption[Área donde la interpolación bilinear y el modelo predictivo difieren en el periodo 06/2014]{Píxeles donde existe una alta diferencia entre la interpolación bilinear con CDO y el modelo predictivo \textit{Random Forest} para junio del año 2014.}
    \label{bilrf2014}
\end{figure}

Finalmente, un estudio comparativo de la variabilidad de las variables explicativas en los parches de altas y bajas diferencias promedio nos puede entregar 
una idea de qué tanto afecta la varianza de cada variable a la predicción. 
\begin{table}[H] 
    \caption[Varianzas de las variables explicativas de RF]{Varianzas de las 6 variables explicativas para los parches con altas diferencias y bajas diferencias promedio entre interpolación mediante CDO y predicción del modelo \textit{Random Forest}}
    \newcolumntype{C}{>{\centering\arraybackslash}X}
    \begin{tabularx}{0.9\textwidth}{CCC}
    \toprule
    \textbf{Variable}	& \textbf{Parche altas diferencias}	& \textbf{Parche bajas diferencias} \\
        \midrule
        \textbf{runoff}		                 & 1.11489$\times 10^{-5}$  & 1.5682$\times 10^{-5}$  \\
        \textbf{soil moisture}		         & 0.0973583                & 0.1291611\\
        \textbf{precipitation}		         & 6.109682                 & 5.185894\\
        \textbf{evapotranspiration}	         & 7.9032$\times 10^{-7}$   & 8.68624 $\times 10^{-7}$\\
        \textbf{canopy water}		         & 9.409$\times 10^{-9}$    & 8.836 $\times 10^{-9}$\\
        \textbf{snow water equivalent}	     & 15.23432                 & 9.04167 \\
        \bottomrule
    \end{tabularx}
\end{table}

Se aprecia que existe mayor varianza en las variables explicativas en en el parche de píxeles donde existe mayor diferencia. Estas grandes diferencias
pueden ser explicadas por este aumento en la varianza de las componentes hidrológicas del modelo.
%
%
% Las varianzas de las 6 variables explicativas para cada 

\section{Discusión}

En base a la gráfica \ref{dsf17} se puede pensar que la estratificación de los datos para entrenar el modelo en la sección 3.4 resulta ser adecuada de acuerdo a las características geográficas del país, donde las anomalías
tienden a la baja en la zona norte, la cual es una zona con alta demanda hídrica pero en general bastante árida, y a la alza en la zona austral, donde existe más precipitación, humedad y menor temperatura.

Si bien las tendencias de GRACE y de las observaciones de pozos contienen cierta correlación lineal, es claro que las zonas en donde se presentan cultivos agrícolas de cualquier índole, existe una tendencia de vaciado de pozos. Además, en zonas donde existe una alta densidad de pozos de bombeo la extracción de agua tiende a ser más consistente y reiterada.

Aunque se pueden encontrar ciertas zonas donde existe una alta varianza en el modelo predictivo de \textit{Random Forest} en relación a la interpolación lineal, si analizamos el marco general, el modelo logra reducir la varianza sin la necesidad de establecer relaciones lineales entre los píxeles vecinos, si no que gracias a
la adición de información de variables hidrológicas que logren representar los comportamientos de las aguas subterráneas, se puede decir de forma casi segura que éstas tienen una incidencia en la predicción modificando el comportamiento de la distribución de GRACE a alta resolución, y cuando éstas presentan una alta variabilidad, es 
posible que esa varianza influya en la predicción de forma favorable, por ejemplo, lograr captar comportamientos insulares del territorio y/o realizar estudios del comportamiento de las aguas subterráneas en el océano, dinámicas que no deberían poseer una relación lineal con las zonas terrestres. CDO sin duda que es un software muy 
bien optimizado que logra realizar operaciones sobre datos grillados de manera práctica y rápida, no obstante, la información adicional que se proporciona 
al \textit{Downscalling} mediante el modelo \textit{Random Forest} no es replicable en CDO, es por esto que es mejor utilizarlo como una herramienta para utilizar en algoritmos más sofisticados. 

Aunque se hayan podido capturar de manera correcta las anomalías de GRACE a alta resolución, no sería problema agregar variables explicativas adicionales al modelo que puedan tener incidencia en los niveles de agua, por ejemplo las observaciones de pozos, aunque éstas no sean regulares en el espacio, se puede
agregar en zonas donde exista una grilla lo suficientemente regular y realizar un aumento de resolución local. Otra variable a considerar puede ser la actividad sismológica del territorio, ya que se ha estudiado que el aumento o disminución en el nivel de aguas subterráneas puede ser considerada como consecuencia de sismos o terremotos \cite{30}, es por esto
que se puede optar por adicionar la profundidad del sismo y su magnitud a la predicción y así se logren obtener resultados más concretos considerando la 
alta actividad sísimica del territorio. Finalmente, adicionar información acerca de la geología presente en el territorio puede dar información de los movimientos o filtraciones que puede poseer alguna capa en especial. 

Existen anomalías en la serie de tiempo de GRACE original que el modelo \textit{Random Forest} no logra capturar, por ejemplo a inicios del período (año 2002), principios del año 2010 y mediados del 2015. Es sabido que Chile es un país con alto índice de sismos, no es una coincidencia que éstas grandes diferencias entre las anomalías ocurran 
en períodos donde han existido actividad sismológica en la cuarta región, como lo son el terremoto de Coquimbo el 18 de junio del 2002 con una magnitud de 6.6, el famoso terremoto ocurrido el 27 de febrero del 2010 que afectó a varias regiones con magnitud de 8.8 y finalmente el terremoto en Coquimbo el 16 de septiembre del 2015 con magnitud 8.4,
es por esta razón que se propone introducir una variable explicativa sismológica para entrenar y predecir las anomalías de GRACE.

Random Forest no es uno de los modelos predictivos más contemporáneo, si bien otorga buena generalización, es posible mejorar los resultados con un modelo basado en árboles más actual, como por ejemplo \texttt{XGBoost} el cual, además de ajustar árboles,
también ajusta los hiperparámetros de forma óptima a través de múltiples herramientas, como la \textbf{paralelización}, el \textit{\textbf{tree pruning}} e inclusive
la \textbf{optimización de hardware}.
% 

%
%

\chapter{Conclusiones}
\label{C6}

La metodología del \textit{Downscalling} propuesta por \cite{11} aunque no fue implementada de inicio a fin, logra entregar buenos resultados a pesar de no poseer un método de validación ortodoxo en la alta resolución, lo cual
complejiza más el trabajo. Aunque la validación tuvo que constar de varias etapas, todas en conjunto presentan las bases de la distribución obtenida con el modelo predictivo.

La extensión de la librería de \texttt{pandas} para el tratamiento y estudio de datos georreferenciados resultó ser una herramienta indispensable para el aumento de resolución, sin la ayuda de \texttt{geopandas} habría sido
mucho más complejo y tardío llebar a cabo el proyecto en cuestión.

La selección, análisis y pre-procesamiento de datos son sin lugar a dudas las tareas más importantes y las que requieren más tiempo, si bien no se escogieron los mismos conjuntos de datos que en la métodología que se discute 
en el capítulo 3, sin lugar a duda los datos que se consiguieron para el territorio chileno además de ser homogéneos, son fiables ya que son proporcionados por centros de climatología especialzados y/o instituciones nacionales, lo cual
es una gran ventaja.

La validación, además de ser una de las tareas más complejas de este proyecto, fue una de las más importantes, ya que se buscaba que las tendencias y comportamientos a gran escala se mantengan a pequeña escala. Las observaciones de pozos
logran corroborar que las tendencias de GRACE a baja y alta resolución coinciden, y aquellas zonas en las que no, los comportamientos poco similares sean causados posiblemente por la explotación de los recursos hídricos o por la adición de variabilidad por parte de las
variables explicativas.

Con los productos grillados obtenidos es posible ampliar aún mas el estudio de las aguas subterráneas en Chile, teniendo un producto más granular los estudios en zonas áridas y zonas que presenten megasequía serán 
más compresibles y particulares, al igual que en las zonas donde aumente el contenido hídrico.

Como es apreciable en este trabajo, en Chile existe un déficit hídrico que cada año aumenta, sobre todo en la zona centro-sur del país. Debemos ser conscientes de lo que está sucediendo hoy, pues son irreparables las consecuencias que trae
la extracción excesiva de las cuencas del territorio y por lo que se aprecia, serán nuestras fuentes de agua más favorables.

%
%
%
%
% ------------------------------------------------------------------------
\bibliography{bibliografia/Bibliografia}

\end{document}