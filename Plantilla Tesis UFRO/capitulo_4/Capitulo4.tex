\chapter{Resultados y Discusión}
\label{C4}
Los resultados debieron ser analizados de dos formas: Una de ellas consiste en la medición del error de cada modelo instanciado en el conjunto de test, estos errores son la 
raíz del error cuadrático medio \ref{eqn:mse}, en inglés \textit{Root Mean Squared Error (RMSE)} y el error medio absoluto (\textit{MAE}) \ref{eqn:mae}, ésto significa una validación en la resolución nativa de 
GRACE, es decir 1$^\circ$. La segunda manera es la validación en la alta resolución, es decir, posterior a realizar la predicción a 0.25$^\circ$; Está claro que no se posee
un conjunto para validar los datos en este escenario debido a que sería un \textit{Downscalling} ya hecho. Por este motivo se deben optar por otras maneras de corroborar la información
para que ésta sea fiable, aquí es donde las series de tiempo se ajustan a lo que buscamos validar.

\section{Métricas de validación en el conjunto de prueba }
Como se ha mencionado, la primera validación ocurre en la resolución nativa de GRACE. Ahora, en primera instancia se evalúa el rendimiento general de cada modelo, es decir, 
se calcula el error medio absoluto y \textit{RMSE} para el conjunto entero de testeo. Con este procedimiento se obtuvieron los siguientes errores:

\begin{itemize}
    \item Para el modelo que \textbf{minimiza el error cuadrático medio}, la métrica \textit{MAE} es de aproximandamente 0,0056904 y un \textit{RMSE} con valor de 6,23$ \times 10^{-5}$.
    \item Para el modelo que \textbf{minimiza el error medio absoluto}, la métrica \textit{MAE} es de 0,00615 aproximadamente y el \textit{RMSE} se obtiene un valor de 7,25$ \times 10^{-5}$.
\end{itemize}

Para ambos modelos, las variables explicativas más importantes fueron \texttt{time}, \texttt{lat}, \texttt{lon}, humedad de suelo (\texttt{sm}) y la evaporación por transpiración de la vegetación (\texttt{e}), en ese orden. Claramente el porcentaje de importancia
varía pero el orden se mantiene. Por lo tanto, se puede afirmar que los modelos se encuentran afectos al rendimiento mayoritariamente en base al criterio de minimización escogido.

La siguiente validación en base a la resolución nativa de GRACE es con respecto a las macrozonas del país, es decir, aquellos datos que fueron estratificados y luego separados en entrenamiento y testeo (figura \ref{traintest}).
En cada macrozona se calculan las métricas de error correspondiente en el conjunto test. Los resultados se muestran en la siguiente tabla:

    \begin{table}[H] 
        \caption[Métricas de validación para cada macrozona]{Métricas de validación para cada macrozona de Chile.}
        \newcolumntype{C}{>{\centering\arraybackslash}X}
        \begin{tabularx}{0.9\textwidth}{CCCCC}
        \toprule
        \textbf{zona}	& \textbf{MAE \texttt{mse\_model}}	& \textbf{MAE \texttt{mae\_model}} & \textbf{RMSE \texttt{mse\_model}} & \textbf{RMSE \texttt{mae\_model}}\\
            \midrule
            \textbf{N. grande}		& 0.004583 & 0.005005 & 0.000043 & 0.000050\\
            \textbf{N. chico}		    & 0.005522 & 0.006020 & 0.000057 & 0.000068\\
            \textbf{Centro}             & 0.006888 & 0.007355 & 0.000084 & 0.000096\\
            \textbf{Sur}                 & 0.005722 & 0.005968 & 0.000061 & 0.000066\\
            \textbf{Austral}            & 0.005963 & 0.006510 & 0.000068 & 0.000082\\
            \bottomrule
        \end{tabularx}
    \end{table}

Para comparar cada modelo en base a las métricas de evaluación, conviene realizar un gráfico, en este caso se realiza uno por cada métrica debido a que el orden de los errores
difieren. 

\begin{figure}[H]
    \centering
          \includegraphics[scale=0.78]{images/RMSE_RFmodels.png}
          \vskip -0.1in
    \caption[Error cuadrático medio en cada macrozona del país]{\footnotesize Medición del error cuadrático medio en cada macrozona de Chile.}
    \label{mserf}
\end{figure}

\begin{figure}[H]
    \centering
          \includegraphics[scale=0.78]{images/MAE_RFmodels.png}
          \vskip -0.1in
    \caption[Error medio absoluto en cada macrozona del país]{\footnotesize Medición del error medio absoluto en cada macrozona de Chile.}
    \label{maerf}
\end{figure}

Es evidente que el modelo el cual minimiza el error cuadrático medio, además de reducir la varianza en la predicción, mejora la predicción misma otorgando un menor error tanto
general como en cada macrozona del país, mostrando la superioridad de la minimización del error cuadrático medio por sobre el error medio absoluto. Es por esta razón, que el modelo escogido para 
predecir valores de \textit{lwe\_thickness} provenientes de GRACE a una mayor resolución será el modelo que mejores métricas de evaluación presente.
%
%
%
%
\section{Resultados Finales}
Es necesario señalar que los resultados que se presentan a continuación son válidos para el territorio continental chileno, debido a la naturaleza de las bases de datos, específicamente de ERA5-Land y precipitación, es 
por este motivo que las observaciones presentes fuera del territorio continental chileno no deben utilizarse para estudios posteriores.

Como se menciona en la metodología del paper base \cite{11}, luego de obtener el TWS final a este se le deben substraer aquellas variables hidrológicas superficiales para así
obtener el contenido de aguas subterráneas, no obstante, no se puede asegurar que realizando la substracción de variables hidrológicas superficiales se consiga el valor del almacenamiento de aguas subterráneas, o 
\textit{Ground Water Storage}, esto debido a muchos otros factores que se ignoran a la hora de realizar el procedimiento, es por esta razón que se decide omitir este paso y proceder a la validación con mediciones \textit{in situ} o mediciones en terreno con datos de pozos de bombeo que 
se detalla más adelante.

Luego de la adición de los residuos interpolados, son 156 las grillas de datos obtenidas desde abril del año 2002 hasta junio del año 2017. A continuación se muestran los datos con reducción de escala para
uno de los primeros períodos de tiempo, específicamente septiembre del año 2002, a mitad del período y para finales del mismo, es decir, abril del año 2017.

\begin{figure}[H]
    \centering
          \subfigure[GRACE a 1$^{\circ}$]{\includegraphics[scale=0.53]{images/Before_Downscalling_v0-09-02.jpeg}}
          \subfigure[TWS a 0.25$^{\circ}$ ]{\includegraphics[scale=0.53]{images/Downscalling_v0-09-02.jpeg}}\goodgap
          \vskip -0.1in
    \caption[\textit{Downscalling} final para el mes 09/2002]{Gráfico del aumento de resolución de GRACE a resolución de 0.25$^{\circ}$ para septiembre del año 2002.}
    \label{dsf02}
\end{figure}


\begin{figure}[H]
    \centering
          \subfigure[GRACE a 1$^{\circ}$]{\includegraphics[scale=0.53]{images/Before_Downscalling_v0-06-14.jpeg}}
          \subfigure[TWS a 0.25$^{\circ}$ ]{\includegraphics[scale=0.53]{images/Downscalling_v0-06-14.jpeg}}\goodgap
          \vskip -0.1in
    \caption[\textit{Downscalling} final para el mes 06/2014]{Gráfico del aumento de resolución de GRACE a resolución de 0.25$^{\circ}$ para junio del año 2014.}
    \label{dsf14}
\end{figure}

A simple vista, se logra apreciar que los mapas de calor son coincidentes en cuanto a los valores de las anomalías se refiere, es decir, el modelo es capaz de capturar correctamente
las zonas en las cuales existe un déficit o superhábit hidrológico en base a las mediciones originales y las variables hidrológicas. Aunque esto parezca concluyente, aún no es suficiente para dterminar si el \textit{Downscalling}
ha sido exitoso, es por esta razón que aún queda la validación de datos de GRACE a alta resolución pendiente.

\begin{figure}[H]
    \centering
          \subfigure[GRACE a 1$^{\circ}$]{\includegraphics[scale=0.53]{images/GRACE_mapv1-04-17.jpeg}}
          \subfigure[TWS a 0.25$^{\circ}$ ]{\includegraphics[scale=0.53]{images/Downscalling_mapv1-04-17.jpeg}}\goodgap
          \vskip -0.1in
    \caption[\textit{Downscalling} final para el mes 06/2014]{Gráfico del aumento de resolución de GRACE a resolución de 0.25$^{\circ}$ para abril del año 2017.}
    \label{dsf17}
\end{figure}



Finalmente, podemos comparar las distribuciones de la resolución nativa de GRACE con el \textit{Downscalling} posterior a la adición de residuos,
utilizando la librería \texttt{seaborn} es poisble obtener dichas densidades, en base a ello se obtienen las gráficas presentes en la figura \ref{distcomp}.
\begin{figure}[H]
    \centering
          \subfigure[Distribución TWS a 1$^{\circ}$]{\includegraphics[scale=0.6]{images/distGRACE1deg.png}}
          \subfigure[Distribución TWS a 0.25$^{\circ}$ ]{\includegraphics[scale=0.6]{images/distGRACE025deg.png}}\goodgap
          \vskip -0.1in
    \caption[Comparativa de distribuciones de GRACE]{Gráfico de las distribuciones de los datos originales de GRACE con los valores predichos a alta resolución con el modelo Random Forest.}
    \label{distcomp}
\end{figure}

Es evidente que el valor de las densiades (eje $y$) difieren en demasía debido al mayor volumen de datos presente en las grillas a alta resolución, de todas maneras, ambas distribuciones
se asemejan y mantienen un coportamiento normal. Ahora, si calculamos ciertos estadísticos útiles, tales como media, desviación estándar y el rango donde se mueve las anomalías a alta y baja resolución podremos tener una idea más clara 
en cuanto a la similitud de distribuciones se trata.

\begin{table}[H] 
    \caption[Comparación de distribuciones de TWS a baja y alta resolución]{Comparación de TWS a baja y alta resolución en distribución, media, desviación estándar y soporte.}
    \newcolumntype{C}{>{\centering\arraybackslash}X}
    \begin{tabularx}{0.9\textwidth}{CCCCC}
    \toprule
    \textbf{resolución}	& \textbf{mínimo}	& \textbf{máximo} & \textbf{media ($\mu$)} & \textbf{desviación estándar ($\sigma$)}\\
        \midrule
        \textbf{baja (1$^{\circ}$)}		& -0.282633 & 0.229261  & 0.000818 & 0.058005\\
        \textbf{alta (0.25$^{\circ}$)}   & -0.284269 & 0.233349	& 0.001885 & 0.057145\\
        \bottomrule
    \end{tabularx}
    \label{distcomp}
\end{table}

De la tabla \ref{distcomp}, se puede deducir que el \textit{Downscalling} ha sido realizado de manera correcta, manteniendo la distribución de los datos originales, disminuyendo ligeramente el rango en el que éstos se encuentran,
aunque con la media desplazada más lejana del origen y con una pequeña disminución en la variabilidad.


%
%
%
%


\section{Estudio y validación con series de tiempo}

\subsection{Zona de estudio}

Debido a la extensa hidrografía presente en el territorio chileno y de los limitados datos de observaciones de pozos, la validación mediante mediciones \textit{in situ} se centrará
en la región de Coquimbo en el periodo donde se puedan comparar las tendencias de GRACE con las observaciones de pozos, es decir, desde abril del año 2002 hasta enero del año 2016. 

En primera instancia, se realiza un estudio general de datos promediados regionales, 
posteriormente se escoge un píxel de baja resolución el cual contiene las observaciones de pozos remuestradas para finalmente estudiar aquellos píxeles a alta resolución que contengan zonas agrícolas \cite{LC}, las cuales necesitan agua para 
poder realizar el regadío. Este es un factor que se tendrá presente a la hora de realizar el estudio de tendencias para GRACE escalado y las observaciones de pozos.

\begin{figure}[H]
    \centering
          \includegraphics[scale=0.6]{images/Wells_with_landcover_coq.jpeg}
          \vskip -0.1in
    \caption[Zona de validación con series de tiempo]{\footnotesize Zona donde se valida el \textit{Downscalling} a través de series de tiempo con las observaciones de pozos medida en metros de profundidad y además las zonas de uso agrícola resaltadas con tonalidad rosa.}
    \label{szcoq}
\end{figure}

\subsection{Validación con series de tiempo}
El estudio tendencial de las series de tiempo comienza de forma general, es decir, conociendo y estudiando las anomalías porcentuales (ecuación \ref{pct_anomalies}) de las series temporales regionales para todo el período de tiempo, para este estudio se consideran 3 variables: En primer lugar las \textbf{anomalías de TWS} de alta y baja resolución promedio, en segundo lugar la 
\textbf{precipitación} promedio y finalmente las \textbf{observaciones de pozos} individuales de la región graficadas en color gris y su promedio mensual.

\begin{figure}[H]
    \centering
          \includegraphics[width=\linewidth]{images/Anomalies_coq_reg.png}
          \vskip -0.1in
    \caption[Series de tiempo para la región de Coquimbo]{\footnotesize Anomalías porcentuales del promedio mensual de la región de Coquimbo.}
    \label{tscoq}
\end{figure}
En la serie temporal se puede apreciar que las tendencias de GRACE original y GRACE escalado son muy similares aunque no del todo; Se puede observar que los cambios porcentuales difieren, sobre todo desde el año 2010 hasta principios de año 2013. Para la precipitación,
es esperable encontrarse con la estacionalidad que presenta para los periodos invernales en el cono sur del globo, además, comienza con una tendencia en descenso y finaliza con un aumento notorio desde principios del 2013, al igual que las anomalías de TWS. Para los datos promediados de pozos, 
se puede observar que la tendencia señala un vaciado constante de los depósitos de aguas subterráneas con un aumento considerable al mediados del año 2015, tendencia la cual se puede explicar con la disminución tendencial del contenido de 
agua terrestre para la región, aún así, la región en general presenta un déficit anual en las aguas subterráneas que es consistente con la localización de las observaciones \textit{in situ} dentro o muy cercanas a las zonas agrícolas. 

En base a la figura \ref{szcoq} es posible identificar aquellas zonas en las cuales existen actividades que requieren un constante y a la vez considerable suministro de agua. Es posible observar las zonas donde existe una alta cantidad de pozos, zonas agrícolas y comparar
tendencialmente las series de tiempo; Para cumplir ese objetivo, se debe identificar el píxel a baja resolución y los píxeles a alta resolución que se encuentren contenidos.

\begin{figure}[H]
    \centering
          \includegraphics[scale=0.6]{images/Pixel1_and_pixels025_to_validate.png}
          \vskip -0.1in
    \caption[Píxeles a 0.25° y 1° que contienen zonas agrícolas y alta densidad de pozos]{\footnotesize Píxeles a 0.25° y 1° que contienen zonas agrícolas y alta densidad de pozos para validar a través de series de tiempo.}
    \label{pixelscoq}
\end{figure}
En la figura \ref{pixelscoq}, se muestra el píxel escogido a baja resolución que posee como centroide las coordenadas (-71.5,-30.5) el cual contiene 27 códicos únicos de pozos, con observaciones en su mayoría dentro de zonas agrícolas. Además, se destacan dos píxeles a alta resolución con centroide en las coordenadas 
(-71.25,-30.75) y (-71.25,-30.5) en color azulado que contienen una densidad de pozos de 9 y 4 respectivamente, por tanto, es posible determinar el impacto
que poseen las zonas agrícolas sobre las mediciones de pozos y comparar el comportamiento tendencial de las anomalías de TWS con el objetivo de dilusidar con más detalle las dinámicas granulares de las variables.


\begin{figure}[H]
    \centering
          \includegraphics[width=\linewidth]{images/Anomalies_pixel-7125S-3075W_0.png}
          \vskip -0.1in
    \caption[Series de tiempo para pixel particular a 1° con pixel a 0.25° contenido]{\footnotesize Anomalías porcentuales para pixel particular a 1° (-71.25, -30.75) con pixel a 0.25° (-71.25,-30.75) contenido.}
    \label{tspixel0}
\end{figure}

\begin{figure}[H]
    \centering
          \includegraphics[width=\linewidth]{images/Anomalies_pixel-7125S-3075W_1.png}
          \vskip -0.1in
    \caption[Series de tiempo para pixel particular a 1° con pixel a 0.25° contenido]{\footnotesize Anomalías porcentuales para pixel particular a 1° (-71.25, -30.75) con pixel a 0.25° (-71.25, -30.5) contenido.}
    \label{tspixel1}
\end{figure}

En cuanto al primer pixel a alta resolución (fig. \ref{tspixel0}), a pesar de que la tendencia de precipitación desde el año 2013 presenta un aumento, no se logra apreciar el impacto que esta tiene sobre las observaciones de GRACE
ya que para el período en cuestión, las anomalías dan a entender la disminución en el contenido de agua terrestre, información que es corroborable con las observaciones de pozos, donde en el píxel a baja resolución se tiene una 
tendencia de vaciado del pozo inclusive cuando en el periodo inicial, es decir, desde el 2002 hasta el 2004 tanto TWS como la profundidad de pozos muestran un aumento en la cantidad de agua. Un comportamiento más inusual sucede con el pozo a alta resolución,
el cual presenta un nivel tendencial constante hasta el comienzo del período de sequía de GRACE (2012) donde el nivel del pozo tiende a disminuir rápidamente hasta el año 2015, para finalmente aumentar en un $120\%$ aproximadamente en tan solo un año. 

Al analizar la figura \ref{tspixel1} la gran diferencia recae en las observaciones de pozos, donde estas presentan una tendencia negativa desde el año 2011, lo que se traduce en una disminución en el nivel de agua, tendencia que se puede
explicar por la sequía presente desde el año 2002 que experimenta GRACE.

Finalmente, en ambas figuras se puede apreciar que las tendencias entre GRACE a 1° se mantienen en cualquiera de los pixel a 0.25°, por lo que el \textit{Downscalling} además de asemejarse en distribución, también lo hace 
en la media temporal de las observaciones.


\subsection{Correlación Temporal}
En cuanto al territorio continental chileno se trata, el coeficiente de correlación definido en la ecuación \ref{eqn:cort} para los datos promediados mensuales entre la resolución original y la resolución objetivo fue de
\textbf{0.6544517} valor cercano a 1, lo cual indica que las tendencias poseen comportamientos similares y cuando una de ellas aumenta, la otra también y viceversa.

Finalmente, para el analisis tendencial general de la cuarta región de Coquimbo, se calcula el índice de correlación de Pearson 
para el promedio de Grace original a baja resolución y los datos con aumento de resolución, ambos normalizados mediante la normalización min-max. El valor del índice 
fue de \textbf{0.65441778}, un poco más bajo que el coeficiente naccional pero es más que aceptable para mencionar que los comportamientos tendenciales del \textit{Downscalling} se mantienen
en base al satélite GRACE original.

%
%
%
%

\section{Inerpolación bilinear y estudio de variabilidad}
El último componente del estudio del \textit{Downscalling} mediante un método de \textit{ensemble} basado en árboles de decisión es la comparación de una interpolación bilinear de los datos de GRACE a la resolución objetivo
con el aumento de resolución realizada mediante el método de Random Forest. Para realizar la interpolación se opta por utilizar el software CDO y como un estudio \textit{a priori} del \textit{Downscalling} se miden los tiempos de remuestreo
espacial de CDO y la predicción realizada por el modelo predictivo en la máquina virtual \texttt{ e2-custom-6-2176}, los tiempos de ejecución fueron de 0.37 y de 4.96 segundos respectivamente.

\begin{figure}[H]
    \centering
          \subfigure[Interpolación bilinear]{\includegraphics[scale=0.53]{images/Downscalling_cdo-09-02.jpeg}}
          \subfigure[TWS a 0.25$^{\circ}$ ]{\includegraphics[scale=0.53]{images/Downscalling_v0-09-02.jpeg}}\goodgap
          \vskip -0.1in
    \caption[Comparación de interpolación bilinear y \textit{Random Forest}]{Gráfico de interpolación bilinear con CDO y la predicción de TWS del modelo \textit{Random Forest} para septiembre del año 2002.}
    \label{interp02}
\end{figure}

Posterior a realizar la interpolación bilinear, los productos grillados obtenidos mediante CDO y el modelo predictivo son promediados por cada píxel en el territorio para así obtener una única grilla que representa los comportamientos temporales promedio. Para la interpolación bilinear se 
obtuvo una varianza de 1,7963$\times 10^{-4}$ y para la regresión 1,3850$\times 10^{-4}$. 

Posterior al cálculo de las varianzas,
se calculan las diferencias absolutas por píxel entre ambas grillas y posteriormente seleccionar $50\%$ de los píxeles que poseen mayor diferencia. Del total de píxeles de la grilla, un $6.7\%$ de los píxeles se encuentran con una mayor diferencia entre el algoritmo predictivo y la interpolación bilinear.
Es posible aislar la zona donde se encuentra el mayor error y poder realizar una comparativa de las dinámicas presentes.

\begin{figure}[H]
    \centering
          \includegraphics[width=0.8\textwidth]{images/area_with_high_differences_cdorf.png}
          \vskip -0.1in
    \caption[Área donde la interpolación bilinear y el modelo predictivo difieren considerablemente]{Píxeles donde existe una alta diferencia (destacados en rojo) entre la interpolación bilinear con CDO y el modelo predictivo \textit{Random Forest}}
    \label{high_diff}
\end{figure}
En la zona delimitada en la figura \ref{high_diff}, se aprecia que en los píxeles donde el modelo difiere de la interpolación es mayoritariamente en donde existe océano. Para obtener un marco de referencia, se escoge el parche en el cual no existen diferencias 
considerables y así estudiar las varianzas de cada variable explicativa, la interpolación bilinear y la prediccón del modelo  \textit{Random Forest}.

\begin{table}[H] 
    \caption[Varianzas en diferentes parches de CDO y RF]{Varianzas en parches con altas y bajas diferencias promedio entre interpolación bilinear con CDO y regresión mediante RF}
    \newcolumntype{C}{>{\centering\arraybackslash}X}
    \begin{tabularx}{0.9\textwidth}{CCC}
    \toprule
    \textbf{Parche}	& \textbf{Varianza CDO}	& \textbf{Varianza \textit{Random Forest}} \\
        \midrule
        \textbf{Altas diferencias}		& 1,1664$\times 10^{-6}$ &3,03805$\times 10^{-6}$  \\
        \textbf{Bajas diferencias}		 & 9,665881$\times 10^{-6}$ & 9,351364 $\times 10^{-6}$\\
        \bottomrule
    \end{tabularx}
\end{table}
\newpage

Es claro que en el parche en cuestión, las anomalías de TWS a baja resolución que entrega el modelo \textit{Random Forest} entrega mucha más variabilidad en la predicción que la interpolación bilinear, pero fuera de ésta zona
la variabilidad del modelo predictivo como de la interpolación bilinear se mantienen cercanas.

\begin{figure}[H]
    \centering
          \subfigure[Interpolación bilinear]{\includegraphics[scale=0.75]{images/rfvbil_2002.png}}
          \subfigure[TWS a 0.25$^{\circ}$ ]{\includegraphics[scale=0.75]{images/bilvrf_2002.png}}
          \vskip -0.1in
    \caption[Área donde la interpolación bilinear y el modelo predictivo difieren en el periodo 09/2002]{Píxeles donde existe una alta diferencia entre la interpolación bilinear con CDO y el modelo predictivo \textit{Random Forest} para septiembre del año 2002.}
    \label{bilrf2002}
\end{figure}

Al momento de observar un mes particular del parche de píxeles de la figura \ref{high_diff} es posible identificar las zonas donde la regresión difiere de la interpolación, por ejemplo para el mes de septiembre del año 2002 (fig. \ref{bilrf2002})
los píxeles oceánicos poseen un valor más bajo en la anomalía de TWS indicando una mayor estabilidad del contenido de agua en esas zonas. El mismo comportamiento también es capturado para el mes de junio del año 2014 (fig. \ref{bilrf2014}).

\begin{figure}[H]
    \centering
          \subfigure[Interpolación bilinear]{\includegraphics[scale=0.75]{images/rfvbil_2014.png}}
          \subfigure[TWS a 0.25$^{\circ}$ ]{\includegraphics[scale=0.75]{images/bilvrf_2014.png}}
          \vskip -0.1in
    \caption[Área donde la interpolación bilinear y el modelo predictivo difieren en el periodo 06/2014]{Píxeles donde existe una alta diferencia entre la interpolación bilinear con CDO y el modelo predictivo \textit{Random Forest} para junio del año 2014.}
    \label{bilrf2014}
\end{figure}

Finalmente, un estudio comparativo de la variabilidad de las variables explicativas en los parches de altas y bajas diferencias promedio nos puede entregar 
una idea de qué tanto afecta la varianza de cada variable a la predicción. 
\begin{table}[H] 
    \caption[Varianzas de las variables explicativas de RF]{Varianzas de las 6 variables explicativas para los parches con altas diferencias y bajas diferencias promedio entre interpolación mediante CDO y predicción del modelo \textit{Random Forest}}
    \newcolumntype{C}{>{\centering\arraybackslash}X}
    \begin{tabularx}{0.9\textwidth}{CCC}
    \toprule
    \textbf{Variable}	& \textbf{Parche altas diferencias}	& \textbf{Parche bajas diferencias} \\
        \midrule
        \textbf{runoff}		                 & 1.11489$\times 10^{-5}$  & 1.5682$\times 10^{-5}$  \\
        \textbf{soil moisture}		         & 0.0973583                & 0.1291611\\
        \textbf{precipitation}		         & 6.109682                 & 5.185894\\
        \textbf{evapotranspiration}	         & 7.9032$\times 10^{-7}$   & 8.68624 $\times 10^{-7}$\\
        \textbf{canopy water}		         & 9.409$\times 10^{-9}$    & 8.836 $\times 10^{-9}$\\
        \textbf{snow water equivalent}	     & 15.23432                 & 9.04167 \\
        \bottomrule
    \end{tabularx}
\end{table}

Se aprecia que existe mayor varianza en las variables explicativas en en el parche de píxeles donde existe mayor diferencia. Estas grandes diferencias
pueden ser explicadas por este aumento en la varianza de las componentes hidrológicas del modelo.
%
%
% Las varianzas de las 6 variables explicativas para cada 

\section{Discusión}

En base a la gráfica \ref{dsf17} se puede pensar que la estratificación de los datos para entrenar el modelo en la sección 3.4 resulta ser adecuada de acuerdo a las características geográficas del país, donde las anomalías
tienden a la baja en la zona norte, la cual es una zona con alta demanda hídrica pero en general bastante árida, y a la alza en la zona austral, donde existe más precipitación, humedad y menor temperatura.

Si bien las tendencias de GRACE y de las observaciones de pozos contienen cierta correlación lineal, es claro que las zonas en donde se presentan cultivos agrícolas de cualquier índole, existe una tendencia de vaciado de pozos, esto como consecuencia del
uso constante requerido de aguas para realizar regadío. Además, en zonas donde existe una alta densidad de pozos de bombeo la extracción de agua tiende a ser más consistente y reiterada.

Aunque se pueden encontrar ciertas zonas donde existe una alta varianza en el modelo predictivo de \textit{Random Forest} en relación a la interpolación lineal, si analizamos el marco general, el modelo logra reducir la varianza sin la necesidad de establecer relaciones lineales entre los píxeles vecinos, si no que gracias a
la adición de información de variables hidrológicas que logren representar los comportamientos de las aguas subterráneas se puede decir de forma casi segura que éstas tienen una incidencia en la predicción modificando el comportamiento de la distribución y cuando éstas presentan una alta variabilidad, es 
posible señalar que esa varianza influye en la predicción como una información adicional al modelo. CDO sin duda que es un software muy 
bien optimizado que logra realizar operaciones sobre datos grillados de manera práctica y rápida, no obstante, la información adicional que se proporciona 
al \textit{Downscalling} mediante el modelo \textit{Random Forest} no es replicable en CDO, es por esto que es mejor utilizarlo como una herramienta para utilizar en algoritmos más sofisticados. 

Aunque se hayan podido capturar de manera correcta las anomalías de GRACE a alta resolución, no sería problema agregar variables explicativas adicionales al modelo que puedan tener incidencia en los niveles de agua, por ejemplo las observaciones de pozos, aunque éstas no sean regulares en el espacio, se puede
agregar en zonas donde exista una grilla lo suficientemente regular y realizar un aumento de resolución local. Otra variable a considerar puede ser la actividad sismológica del territorio, por ejemplo la profundidad del sismo y su magnitud, es probable que se logren obtener resultados aún más concretos considerando la 
alta actividad sísimica del territorio. Finalmente, adicionar información acerca de la geología presente en el territorio puede dar información de los movimientos o filtraciones que puede poseer alguna capa en especial. 

Existen anomalías en la serie de tiempo de GRACE original que el modelo \textit{Random Forest} no logra capturar, por ejemplo a inicios del período (año 2002), principios del año 2010 y mediados del 2015. Es sabido que Chile es un país con alto índice de sismos, no es una coincidencia que éstas grandes diferencias entre las anomalías ocurran 
en períodos donde han existido actividad sismológica en la cuarta región, como lo son el terremoto de Coquimbo el 18 de junio del 2002 con una magnitud de 6.6, el famoso terremoto ocurrido el 27 de febrero del 2010 que afectó a varias regiones con magnitud de 8.8 y finalmente el terremoto en Coquimbo el 16 de septiembre del 2015 con magnitud 8.4,
es por esta razón que se propone introducir una variable explicativa sismológica para entrenar y predecir las anomalías de GRACE. No se debe descartar que ésta información adicional pueda generar mayor correlación lineal entre las series de tiempo.

Random Forest no es uno de los modelos predictivos más contemporáneo, si bien otorga buena generalización, es posible mejorar los resultados con un modelo basado en árboles más actual, como por ejemplo \texttt{XGBoost} el cual, además de ajustar árboles,
también ajusta los hiperparámetros de forma óptima a través de múltiples herramientas, como la \textbf{paralelización}, el \textit{\textbf{tree pruning}} e inclusive
la \textbf{optimización de hardware}.
% 

%
%