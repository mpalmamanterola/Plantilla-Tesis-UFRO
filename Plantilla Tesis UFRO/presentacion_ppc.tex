\documentclass{beamer}
\usepackage[utf8]{inputenc}
\usepackage{outlines}
\usepackage{bookmark}
\usepackage{amsmath,amssymb,lmodern}

\usetheme{Warsaw}
\usecolortheme{default}

%------------------------------------------------------------
%This block of code defines the information to appear in the
%Title page
\title[Aumento de resolución a través del modelo \textit{Random Forest}] %optional
{Reducción de escala de datos satelitales}

\subtitle{a través del modelo de Bosque Aleatorio}

\author[] % (optional)
{Matías Palma Manterola}

\institute[UFRO] % (optional)
{
Universidad de La Frontera
}

\date[] % (optional)
{Octubre, 2022}

\logo{\includegraphics[height=1.3cm]{images/ufro.jpg}}

\AtBeginSection[]
{
  \begin{frame}
    \frametitle{Tabla de Contenidos}

    \tableofcontents[currentsection]
  \end{frame}
}
%------------------------------------------------------------


\begin{document}

%The next statement creates the title page.
  \frame{\titlepage}

  \section{Introducción}

  \subsection*{Contexto y descripción del problema}

  \begin{frame}
    \frametitle{Contexto y descripción del problema}

    \begin{columns}

      \begin{column}{0.5\textwidth}
        \begin{outline}
          \1 {Almacén de aguas subterráneas.}
            \2 Calidad.
            \2 Cantidad.
            \2 Ampliamente distribuidas.
          \1{Sobrexplotación.}
            \2 Disminución del nivel.
          \1 {Monitoreo de cuencas.}
          \1 {Satélite GRACE.}
            \2 Estimación del agua total presente.
        \end{outline}
      \end{column}

      \begin{column}{0.5\textwidth}
        \begin{center}
          \includegraphics[width=0.9\textwidth]{images/sequía.png}
        \end{center}
      \end{column}

    \end{columns}

  \end{frame}

  \subsection*{GRACE}

  \begin{frame}
    \frametitle{Satélite GRACE}

    \begin{columns}

      \begin{column}{0.6\textwidth}
        \begin{outline}
          \1 \textit{Gravity Recovery And Climate Experiment.}
          \1 Satélites que orbitan sobre la Tierra.
            \2 200 Km de distancia entre ellos.
          \1 Captar movimientos de las aguas.
          \2 Tiene como objetivo contrarrestar:
            \3 Sequías. 
            \3 Inundaciones.
            \3 Socavones.
          \1 Mediciones a baja resolución.
        \end{outline}
      \end{column}

      \begin{column}{0.5\textwidth}
        \begin{center}
          \includegraphics[width=0.9\textwidth]{images/GRACE.jpg}
        \end{center}
      \end{column}

    \end{columns}

  \end{frame}

  \begin{frame}
    \frametitle{Anomalías \textit{TWS}}

    \begin{outline}
      \1 \textit{Terrestrial Water Storage}
      \1 Agua almacenada sobre y debajo de la superficie.
        \2 Agua de dosel.
        \2 Ríos y lagos.
        \2 Humedad de suelo.
        \2 Aguas subterráneas.
    \end{outline}

    \begin{block}{Anomalía TWS}
      $$ TWS_t = \frac{TWS_t - \mu}{\delta},~~~~ \mu, \delta \in \mathbb{R}$$
    \end{block}

  \end{frame}

  \subsection*{Objetivos}

  \begin{frame}
    \frametitle{Objetivo general y específicos}

    \large\textbf{Objetivo general}
    \begin{outline}
      \1 Implementar un aumento de resolución a datos satelitales proporcionados por GRACE a través de un modelo
      predictivo para obtener productos grillados en Chile a alta resolución.
    \end{outline}

    \large\textbf{Objetivos específicos} 
    \begin{outline}
      \1 Definir la metodología.
      \1 Recolectar datos necesarios.
      \1 Preprocesar los datos.
      \1 Implementar y entrenar el modelo predictivo.
      \1 Validar predicciones en base a observaciones \textit{in situ}.
    \end{outline}

  \end{frame}

  \section{Fundamentos Teóricos}

  \subsection*{Modelo del bosque aleatorio}

  \begin{frame}
    \frametitle{\textit{Random Forest}}

    \begin{outline}
      \1 
    \end{outline}
  \end{frame}

  \subsection*{Indice de Correlación de Pearson}

  \begin{frame}
    \frametitle{Indice de Correlación de Pearson}

    \begin{outline}
      \1 Busca cuantificar la similitud lineal en las tasas de crecimiento.
    \end{outline}

    \begin{block}{Indice de Correlación}
      Sean $ S_1 $ y $ S_2 $ dos series temporales con observaciones $ u_1,\cdots, u_p $ y $ v_1,\cdots,v_p $, respectivamente, entonces
      $$ R^2(S_1,S_2) = \frac{\sum_{j=1}^p\sum_{i=1}^p(u_i-u_j)(v_i-v_j)} {\sqrt{\sum_{i=1}^p(u_i-u_j)^2}\sqrt{\sum_{i=1}^p(v_i-v_j)^2}} \in [-1,1]$$
    \end{block}

  \end{frame}

  \section{Metodología}
  \subsection*{Metodología propuesta}

  \begin{frame}
    \frametitle{}
  \end{frame}

  \subsection*{Recolección y preprocesamiento de datos}

  \begin{frame}
    \frametitle{}
  \end{frame}

  \subsection*{Entrenamiento y testeo}

  \begin{frame}
    \frametitle{}
  \end{frame}

  \section{Resultados}

  \subsection*{Métricas de validación}

  \begin{frame}
    \frametitle{}
  \end{frame}

  \subsection*{Productos finales}

  \begin{frame}
    \frametitle{}
  \end{frame}

  \subsection*{Validación de datos}

  \begin{frame}
    \frametitle{}
  \end{frame}

  \section{Conclusiones}

  \begin{frame}
    \frametitle{}
  \end{frame}

\end{document}