\documentclass{beamer}
\usepackage[utf8]{inputenc}
\usepackage[spanish]{babel}
\usepackage{outlines}
\usepackage{bookmark}
\usepackage{graphicx}
\usepackage{tabularx}
\usepackage[font=tiny]{caption}
\usepackage[export]{adjustbox}
\setbeamertemplate{caption}[numbered]
\usepackage{amsmath,amssymb,lmodern}
\usepackage{hyperref} 
\usepackage{xcolor}
\usetheme{Warsaw}
\usecolortheme{default}

%------------------------------------------------------------
%This block of code defines the information to appear in the
%Title page
\title[Aumento de resolución a través del modelo \textit{Random Forest}] %optional
{Aumento de resolución de datos}

\subtitle{a través de métodos de Machine Learning  }

\author[] % (optional)
{Matías Palma Manterola}

\institute[UFRO] % (optional)
{
  Universidad de La Frontera
}

\date[] % (optional)
{\today}



\logo{\includegraphics[height=1cm]{images/ufro.png}}

\AtBeginSection[]
{
  \begin{frame}
    \frametitle{Tabla de Contenidos}

    \tableofcontents[currentsection]
  \end{frame}
}
%------------------------------------------------------------


\begin{document}

%The next statement creates the title page.
  \frame{\titlepage}

  \section{Introducción}

  \subsection*{Contexto y descripción del problema}

  \begin{frame}
    \frametitle{Contexto y descripción del problema}

    \begin{columns}

      \begin{column}{0.5\textwidth}
        \begin{outline}
          \1 {Almacén de aguas subterráneas.}
            \2 Calidad.
            \2 Cantidad.
            \2 Ampliamente distribuidas.
          \1{Sobrexplotación.}
            \2 Disminución del nivel.
          \1 {Monitoreo de cuencas.}
          \1 {Satélite GRACE.}
            \2 Estimación del agua total presente.
        \end{outline}
      \end{column}

      \begin{column}{0.5\textwidth}
        \begin{center}
          \includegraphics[width=0.9\textwidth]{images/sequía.png}
        \end{center}
      \end{column}

    \end{columns}

  \end{frame}

  \subsection*{GRACE}

  \begin{frame}
    \frametitle{Satélite GRACE}

    \begin{columns}

      \begin{column}{0.6\textwidth}
        \begin{outline}
          \1 \textit{Gravity Recovery And Climate Experiment.}
          \1 Satélites que orbitan sobre la Tierra.
            \2 200 Km de distancia entre ellos.
          \1 Captar movimientos de las aguas.
          \2 Contrarrestar:
            \3 Sequías. 
            \3 Inundaciones.
            \3 Socavones.
          \1 Mediciones a baja resolución.
        \end{outline}
      \end{column}

      \begin{column}{0.5\textwidth}
        \begin{figure}
          \begin{center}
            \includegraphics[width=0.9\textwidth]{images/GRACE.jpg}
          \end{center}
          \caption{\tiny{Misión GRACE lanzada el año 2002.}}
        \end{figure}
      \end{column}

    \end{columns}

  \end{frame}

  \begin{frame}
    \frametitle{Anomalías \textit{TWS}}

    \begin{outline}
      \1 \textit{Terrestrial Water Storage}
      \1 Agua almacenada sobre y debajo de la superficie.
        \2 Agua de dosel.
        \2 Ríos y lagos.
        \2 Humedad de suelo.
        \2 Aguas subterráneas.
    \end{outline}

    \begin{block}{Anomalía TWS}
      $$ TWS_t = \frac{TWS_t - \mu}{\delta},~~~~ \mu, \delta \in \mathbb{R}$$
    \end{block}

  \end{frame}

  \subsection*{Reducción de escala}

  \begin{frame}
    \frametitle{\textit{Downscaling}}

    \begin{columns}

      \begin{column}{0.4\textwidth}
        \begin{outline}
          \1 Inferir información de alta resolución.
            \2 Datos satelitales.
            \2 Imágenes.
          \1 Enfoques dinámicos o estadísticos.
            \2 Meteorología.
            \2 Climatología
            \2 Teledetección
        \end{outline}
      \end{column}
      
      \begin{column}{0.6\textwidth}
        \begin{figure}
          \begin{center}
            \includegraphics[width=\textwidth]{images/downscaling_example.png}
          \end{center}
          \caption{Reducción de escala.}
        \end{figure}
      \end{column}
    \end{columns}
  \end{frame}

  \subsection*{Objetivos}

  \begin{frame}
    \frametitle{Objetivo general y específicos}

    \large\textbf{Objetivo general}
    \begin{outline}
      \1 Implementar un aumento de resolución a datos satelitales proporcionados por GRACE a través de un modelo
      predictivo para obtener productos grillados en Chile a alta resolución.
    \end{outline}

    \large\textbf{Objetivos específicos} 
    \begin{outline}
      \1 Definir la metodología.
      \1 Recolectar datos necesarios.
      \1 Preprocesar los datos.
      \1 Implementar y entrenar el modelo predictivo.
      \1 Validar predicciones en base a observaciones \textit{in situ}.
    \end{outline}

  \end{frame}

  \section{Fundamentos Teóricos}

  \subsection*{Píxeles y productos grillados}

  \begin{frame}
    \frametitle{Productos grillados}

    \begin{block}{Píxel}
      \begin{outline}
        \1 Unidad básica de una imágen.
        \1 Cada píxel contiene canales de información
          \2 RGB.
          \2 Números reales ($\mathbb{R}$).
          \2 Eiquetas de clase.
      \end{outline}
    \end{block}

    \begin{block}{Ráster}
      \begin{outline}
        \1 Conjunto de píxeles ordenados.
          \2 Filas.
          \2 Columnas.
        \1 El valor del pixel se asocia al centroide del mismo.
      \end{outline}
    \end{block}
  \end{frame}


  \subsection*{Modelo del bosque aleatorio}

  \begin{frame}
    \frametitle{Árboles de decisión}

    \begin{outline}
      \1  Algoritmo de aprendizaje automático del tipo supervisado.
      \1 Segmentación recursiva del espacio de predicción en regiones regulares.
        \2 Media. 
        \2 Moda.
      \1 Minimiza la suma de cuadrados de cada región (regresión).
    \end{outline}
  \end{frame}

  \begin{frame}
    \frametitle{\textit{Random Forest}}

    \begin{outline}
      \1 Conjunto de árboles de decisión.
        \2 \textit{Bootstrap sample}.
      \1 Contrarresta la inestabilidad de los árboles.
    \end{outline}
  \end{frame}

  \subsection*{Indice de Correlación de Pearson}

  \begin{frame}
    \frametitle{Indice de Correlación de Pearson}

    \begin{outline}
      \1 Busca cuantificar la similitud lineal en las tasas de crecimiento.
    \end{outline}

    \begin{block}{Indice de Correlación}
      Sean $ S_1 $ y $ S_2 $ dos series temporales con observaciones $ u_1,\cdots, u_p $ y $ v_1,\cdots,v_p $, respectivamente, entonces
      $$ R^2(S_1,S_2) = \frac{\sum_{j=1}^p\sum_{i=1}^p(u_i-u_j)(v_i-v_j)} {\sqrt{\sum_{i=1}^p(u_i-u_j)^2}\sqrt{\sum_{i=1}^p(v_i-v_j)^2}} \in [-1,1]$$
    \end{block}

  \end{frame}

  \section{Metodología}
  \subsection*{Metodología propuesta}

  \begin{frame}
    \frametitle{Metodología adoptada}

    \begin{figure}
      \begin{center}
        \includegraphics[height=0.835\textheight]{images/metodologia_Chen_etal2019.png}
        \caption{Metodología utilizada en (Chen \textit{et al.}, 2019), pág. 7}
      \end{center}
  \end{figure}
    
  \end{frame}

  \subsection*{Recolección y preprocesamiento de datos}

  \begin{frame}
    \frametitle{Dominio espacio-temporal}

    \begin{columns}

      \begin{column}{0.5\textwidth}
        \begin{outline}
          \1 Territorio continental chileno.
            \2 Longitud desde 66.5°E hasta 77.5°E.
            \2 Latitud 16.5°S hasta 56.75°S
          \1 Aumento de resolución de productos grillados desde abril del año 2002 hasta febrero del año 2017.
        \end{outline}
      \end{column}

      \begin{column}{0.5\textwidth}
        \begin{figure}
          \centering
          \includegraphics[height=0.7\textheight]{images/limites_regionales.png}
          \caption{Datos de grace para abril del año 2002 en el territorio continental chileno.}
        \end{figure}
      \end{column}

    \end{columns}

  \end{frame}

  \begin{frame}
    \frametitle{GRACE TWS}
    \begin{columns}

      \begin{column}{0.5\textwidth}
        \begin{outline}
          \1 Anomalías TWS.
          \1 Datos globales mensuales.
          \1 Resolución a 1° ($\sim 110$ Km).
          \1 \texttt{lwe\_thickness}: grosor de agua líquida equivalente.
        \end{outline}
      \end{column}

      \begin{column}{0.5\textwidth}
        \begin{figure}
          \centering
          \includegraphics[height=0.75\textheight]{images/grace_crudo.png}
          \caption{Datos obtenidos desde la base de datos 
            \href{https://podaac-opendap.jpl.nasa.gov/opendap/allData/tellus/L3/grace/land_mass/RL06/v04/CSR/}
            {\textcolor{blue}{\underline{OP$e$NDAP}}.}}
        \end{figure}
      \end{column}

    \end{columns}
  \end{frame}

  \begin{frame}
    \frametitle{ERA5-Land}
    \begin{columns}

      \begin{column}{0.5\textwidth}
        \begin{outline}
          \1 Representa la evolución de variables terrestres.
            \2 Humedad de suelo.
            \2 Nieve equivalente en agua.
            \2 Evapotranspiración.
            \2 Agua de dosel.
            \2 Escorrentía.
          \1 Combina datos de modelos con observaciones.
          \1 Datos globales mensuales
          \1 Resolución de 0.1° ($\sim$ 9 Km)
        \end{outline}
      \end{column}

      \begin{column}{0.5\textwidth}
        \begin{figure}
          \centering
          \includegraphics[width=\textwidth]{images/era5.jpg}
          \caption{Temperatura promedio mensual obtenida de la base de datos 
          \href{https://cds.climate.copernicus.eu/cdsapp\#!/dataset/reanalysis-era5-land-monthly-means?tab=form/}
          {\textcolor{blue}{\underline{Copernicus}}.}}
        \end{figure}
      \end{column}

    \end{columns}
  \end{frame}

  \begin{frame}
    \frametitle{(CR)$^2$-met}
    \begin{columns}
      \begin{column}{0.5\textwidth}
        \begin{outline}
          \1 Información meteorológica
            \2 Precipitación.
            \2 Temperatura.
          \1 Datos muestreados diariamente para el territorio continental.
          \1 Modelos estadísticos para traducir diversos componentes.  
          \1 Resolución de 0.05° ($\sim$ 5 Km)
        \end{outline}
      \end{column}

      \begin{column}{0.5\textwidth}
        \begin{figure}
          \centering
          \includegraphics[height= 0.6\textheight]{images/cr2met.png}
          \caption{Precipitación acumulada anual promedio provenientes de la base de datos 
            \href{https://www.cr2.cl/datos-productos-grillados/}{\textcolor{blue}{\underline{(CR)$^2$-met}}.}}
        \end{figure}
      \end{column}
    \end{columns}

  \end{frame}

  \begin{frame}
    \frametitle{Dirección General de Aguas}

    \begin{columns}
    
      \begin{column}{0.5\textwidth}
        \begin{outline}
          \1 Observaciones de pozos de bombeo.
          \1 Mediciones diarias para la zona centro y norte del país.
          \1 Mide profundidad (m) del pozo.
            \2 Mayor profundidad implica menor cantidad de agua disponible.
        \end{outline}
      \end{column}

      \begin{column}{0.5\textwidth}
        \begin{figure}
          \centering
          \includegraphics[height=0.65\textheight]{images/Observacion_pozos.jpeg}
          \caption{Observaciones de pozos provenientes de la DGA proporcionados por (CR)$^2$.}
        \end{figure}
      \end{column}

    \end{columns}
  \end{frame}


  \subsection*{Entrenamiento y testeo}

  \begin{frame}
    \frametitle{Conjunto de entrenamiento y de prueba}

    \begin{columns}

      \begin{column}{0.5\textwidth}
        \begin{outline}
          \1 Alta variabilidad climatológica en el territorio.
          \1 Estratificación de datos.
            \2 Disminución de sesgo.
            \2 Predicciones coherentes respecto a la naturaleza de las variables.
        \end{outline}
      \end{column}

      \begin{column}{0.5\textwidth}
        \begin{figure}
          \begin{center}
            \includegraphics[height=0.7\textheight]{images/macrozonas.png}
          \end{center}
          \caption{Macrozonas de Chile.}
        \end{figure}
      \end{column}

    \end{columns}
  \end{frame}  

  \section{Resultados}

  \subsection*{Métricas de validación}

  \begin{frame}
    \frametitle{Métricas de validación}

    \begin{center}
      \begin{tabular}{| c | c | c |}
        \hline
        Modelo/Métrica de eval. & \parbox{7em}{\centering Error medio \\asboluto (MAE)} & \parbox{7em}{\centering Raíz del error \\ cuadrático medio (RMSE)}\\ \hline
        Mín. error cuadrático medio &  \textcolor{olive}{$5.69\times 10^{-3}$} & \textcolor{olive}{$6.23 \times 10^{-5}$}\\
        Mín. error medio absoluto & \textcolor{gray}{$6.15\times 10^{-3}$} & \textcolor{gray}{$7.25 \times 10^{-5}$}\\
        \hline
      \end{tabular}
      
    \end{center}

    \begin{columns}

      \begin{column}{0.5\textwidth}
        \begin{figure}
          \centering
          \includegraphics[width=\textwidth]{images/MAE_RFmodels.png}
          \caption{Error medio absoluto}
        \end{figure}
      \end{column}

      \begin{column}{0.5\textwidth}
        \begin{figure}
          \centering
          \includegraphics[width=\textwidth]{images/RMSE_RFmodels.png}
          \caption{Raíz del error cuadrático medio}
        \end{figure}
      \end{column}
    \end{columns}
  \end{frame}

  \subsection*{Productos finales}

  \begin{frame}
    \frametitle{}

    \begin{columns}
      \begin{column}{0.5\textwidth}

        \begin{figure}

        \begin{columns}
          \begin{column}{0.5\textwidth}
            \includegraphics[height=0.75\textheight, right]{images/Before_Downscalling_v0-09-02.jpeg}
          \end{column}

          \begin{column}{0.5\textwidth}
            \centering
            \includegraphics[height=0.75\textheight, left]{images/Downscalling_v0-09-02.jpeg}
          \end{column}
        \end{columns}

        \caption{Productos grillados a alta y baja resolución para el mes de septiembre del año 2002.}
        \end{figure}
      \end{column}

      \begin{column}{0.5\textwidth}

        \begin{figure}

          \begin{columns}

            \begin{column}{0.5\textwidth}
              \includegraphics[height=0.75\textheight, right]{images/GRACE_mapv1-04-17.jpeg}
            \end{column}

            \begin{column}{0.5\textwidth}
              \centering
              \includegraphics[height=0.75\textheight, left]{images/Downscalling_mapv1-04-17.jpeg}
            \end{column}

          \end{columns}

          \caption{Productos grillados a alta y baja resolución para el mes de abril del año 2017.}
        \end{figure}
      \end{column}
    \end{columns}
  \end{frame}

  \subsection*{Validación de datos}

  \begin{frame}
    \frametitle{Zona de validación}
    \begin{figure}
      \centering
      \includegraphics[height=0.8\textheight]{images/Wells_with_landcover_coq.jpeg}
      \caption{Región de Coquimbo con observaciones de pozos y zonas agrícolas.}
    \end{figure}

  \end{frame}

  \begin{frame}
    \frametitle{Series de tiempo}
    
    \centering
    \includegraphics[width=\textwidth]{images/Anomalies_coq_reg.png}
  
  \end{frame}

  \begin{frame}
    \begin{figure}
      \centering
      \includegraphics[height=0.9\textheight]{images/Pixel1_and_pixel025.png}
      \caption{Píxel a alta y baja resolución para validar en zpnas agrícolas con observaciones de Pozos.}
    \end{figure}
  \end{frame}

  \begin{frame}
    \frametitle{Series de tiempo}
    
    \centering
    \includegraphics[width=\textwidth]{images/Anomalies_pixel-7125S-3075W_0.png}
  \end{frame}



  \subsection*{Estudio de variabilidad}

  \begin{frame}
    \frametitle{Interpolación bilinear y Bosque aleatorio}

    \begin{outline}
      \1 Diferencias absolutas entre promedio de píxeles.
        \2 Selección de píxeles con una diferencia mayor al 50\%.
          \3 6.7\% del total de píxeles.
      \1 Aislamiento de la zona en cuestión.
    \end{outline}
  \end{frame}

  \begin{frame}
    \begin{columns}

      \begin{column}{0.5\textwidth}
        \begin{figure}
          \centering
          \includegraphics[width=\textwidth]{images/altasdiferencias.png}
          \caption{Parche a alta resolución donde la interpolación bilinear y la regresión usando \textit{Random Forest} difieren en sobremanera.}
        \end{figure}
      \end{column}

      \begin{column}{0.5\textwidth}
        \begin{figure}
          \centering
          \includegraphics[width=\textwidth]{images/bajasdiferencias.png}
          \caption{Parche a alta resolución donde la interpolación bilinear y la regresión usando \textit{Random Forest} no difieren en demasía.}
        \end{figure}
      \end{column}

    \end{columns}
  \end{frame}

  \begin{frame}
    \frametitle{Estudio de variabilidad}

    \begin{center}
      \begin{tabular}{|c|c|c|}
        \hline
        \textbf{Parche}	& \textbf{$\sigma^2$ CDO}	& \textbf{$\sigma^2$ \textit{Random Forest}} \\
        \hline
        \textbf{Altas diferencias}		& 1,1664$\times 10^{-6}$ &3,03805$\times 10^{-6}$  \\
        \textbf{Bajas diferencias}		 & 9,665881$\times 10^{-6}$ & 9,351364 $\times 10^{-6}$\\
        \hline
      \end{tabular}
    \end{center}
  \end{frame}

  \begin{frame}
    \frametitle{Estudio de variabilidad}
    \centering
    \begin{tabular}{|c|c|c|}
      \hline
      \textbf{Variable}	& \textbf{Altas diferencias}	& \textbf{Bajas diferencias} \\
      \hline
      \textbf{Soil moisture}		         & 0.0973583                  & 0.1291611\\
      \textbf{Evapotranspiration}	       & 7.9032 $\times 10^{-7}$    & 8.68624 $\times 10^{-7}$\\
      \textbf{Runoff}		                 & 1.11489$\times 10^{-5}$    & 1.5682 $\times 10^{-5}$  \\
      \textbf{Precipitation}		         & 6.109682                   & 5.185894\\
      \textbf{Canopy water}		           & 9.409  $\times 10^{-9}$    & 8.836  $\times 10^{-9}$\\
      \textbf{Snow water equivalent}	   & 15.23432                   & 9.04167 \\
      \hline
      \textbf{$\sigma^2$ promedio}         & 3.57356                    & 2.39279\\
      \hline
    \end{tabular}
  \end{frame}

  \section{Conclusiones}

  \begin{frame}
    \frametitle{Conclusiones}

    \begin{outline}
      \1 La metodología adoptada logra entregar predecir de forma válida los resultados.
      \1 Las variables explicativas son datos homogéneos y fiables de centros especializados.
      \1 La validación de resultados mostró una alta variabilidad en zonas donde las variables explicativas también la presentaban.
      \1 Con los productos grillados obtenidos es posible estudiar con más detalle y de manera más granular el comportamiento
      de las aguas subterráneas en Chile.
      
    \end{outline}
  \end{frame}

\end{document}