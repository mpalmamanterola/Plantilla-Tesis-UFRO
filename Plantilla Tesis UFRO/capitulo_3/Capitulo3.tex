\chapter{Metodología}
\label{C3}
Para cumplir el objetivo general se utilizó el lenguaje de programación Python debido a que es un lenguaje de \textbf{alto nivel}, por tanto es más fácil de utilizar como de leer debido a su sintaxis simple.
Además, Python resulta ser un lenguaje \textbf{polivalente multiparadigma} lo que hace que sea de propósito general, en particular es una excelente herramienta para \textit{Data Science}. Otro aspecto importante es la
\textbf{portabilidad}, la cual permite que el código no deba ser modificable de ninguna manera para ejecutarse en cualquier sistema operativo (macOS, Linux, UNIX y Windows) y finalmente que es un lenguaje
\textbf{gratis y de código abierto} lo que permite que pueda ser usado, modificado y distribuido por cualquier persona.

Tras un estudio exhaustivo de referencias y documentación se decide optar por aquella metodología
que se acercase más a la problemática local con datos que puedan ser obtenidos fácilmente. En base a esto se escoge el paper \textit{Downscaling of GRACE-Derived Groundwater Storage
Based on the Random Forest Model} \cite{11}. 

Otra de las herramientas utilizadas durante todo el proceso fue la Máquina Vritual (VM por sus siglas en inlés \textit{Virtual Machine}) creada en 
\textit{Google Cloud Plataform (gcp)} y accediendo a ella mediante una conexión \textit{ssh} a través del perfil de \textit{GitHub}. En pocas palabras el proceso consiste, en primer lugar, crear las máquinas virtuales;
En este caso se generaron 2 VM con diferentes características y funciones, la primera está destinada para el procesamiento más extenso y que necesite una alta capacidad de procesamiento para llevar a cabo las tareas más complejas en menos tiempo,
así, las características de la máquina se resumen en el código \texttt{e2-highmem-16} que se traduce como un hardware de una CPU de 16 núcleos virtuales de 128 GB de memoria RAM y la segunda VM es la que se utilizará
más constantemente con una capacidad menor pero que sea capaz de soportar el nivel de cálculo necesario, las carácterísticas de la máquina codifcadas son \texttt{e2-custom-6-21760} que consta de 6 núcleos virtuales,
2 físicos y aproximadamente 21 GB de RAM. Ambas máquinas virtuales utilizan un disco persistente balanceado de 100GB. 

Luego de haber generado los entornos en los cuales se procesarán y guardarán los datos, se crea el repositorio en \textit{GitHub} en 
donde se irá guardando y archivando el código generado para realizar el \textit{Downscaling}. La ventaja de utilizar \textit{git} es que proporciona un espacio seguro y con respaldo ante cualquier eventualidad.

\begin{figure}[H]\label{sshconec}
    \centering
          \includegraphics[width=0.6\linewidth]{images/ssh-conection.png}
    \caption[Conexión ssh de máquina virtual]{\footnotesize Conexión ssh de máquina virtual ``matias-exploracion-5'' (\texttt{e2-custom-6-21760}) con el \textit{e-mail} ``m.palma14'' asociado usuario de \textit{GitHub} mediante el archivo \textit{google-connect-docker.sh} en LINUX.}
\end{figure}

La tercera parte consiste en configurar los datos globales de nombre de usuario y \textit{e-mail} en \textit{git} para
enlazar a la conexión, posteriormente crear una llave de conexión \textit{ssh} y agregarla al repositorio creado en \textit{GitHub}, para lograr conectarse a cualquiera de las máquinas virtuales que se deseen utilizar y así poder realizar modificaciones 
al repositorio y registrarlas mediante \textit{commits}.
Luego de haber activado la máquina virtual en \textit{gcp} que se desea utilizar se procede a conectarla (fig. \ref{sshconec}.1) con la dirección IP del computador personal 
mediante un archivo con extensión \textit{.sh} el cual solicita la información del usuario y de la máquina y la coteja.

Para finalizar, lo único que resta es elegir el puerto al cual el usuario desea conectarse en el navegador ya que posee múltiples entornos de ejecución de código Python, entre ellos \textit{Jupyter Notebook}, \textit{Jupyter Lab} y \textit{Dask Dashboard}
los cuales entregan diferentes herramientas y formatos para realizar el Análisis de Datos. Para este trabajo se optó por utilizar \textit{Jupyter Lab} ya que es aquel que permite tener un control más detallado del directorio de archivos, acceso a la terminal
de la VM de forma rápida, lo cual aumenta su eficiencia a la hora de trabajar con la máquina remota. 


%
%
%
%
\section{Métodología propuesta en el artículo}

\begin{figure}[H]
    \centering
          \includegraphics[scale=0.7]{images/metodología_Chen_etal2019.png}
          \vskip -0.1in
    \caption[Metodología de reducción de dimensión propuesta]{\footnotesize Metodología propuesta en \cite{11}, pág. 7.}
    \label{metodologia}
\end{figure}

En base a \cite{11}, se procede a replicar la metodología propuesta en la figura 3.1, donde se detalla
la forma en la que los autores realizan la reducción de escala a una resolución inicial de $1^{\circ} (\sim 111 \text{ Km})$
para obtener datos georreferenciados a una resolución de 0.25$^{\circ}~(\sim 27 \text{.} 75 \text{ Km})$ en base a 4 diferentes fuentes de datos los cuales se encuentran a una alta resolución: 
\textit{GRACE} para el \textit{Terrestrial Water Sotrage}, \textit{GLDAS} para 4 diferentes variables hidrológicas (\textit{canopy water}, \textit{soil moisture}, 
\textit{snow water equivalent} y \textit{runoff}), \textit{GLEAM} para datos 
de vapor de agua emanado por plantas, el suelo, etc, conocida como \textit{evapotranspiration} y \textit{TRAMM} para datos de precipitación.

La premisa del procedimiento se basa en que las variables explicativas se encuentran a una resolución mucho mayor que los datos recopilados por GRACE,
por lo que en base al comportamiento de cada una de ellas es posible determinar las dinmámicas de la reserva de agua terrestre en puntos donde el satélite no 
logra captar anomalías más granulares.

El proceso de reducción de escala que se detalla en la figura \ref{metodologia}, en pocas palabras se puede describir
como:
\begin{itemize}

    \item Las 6 variables explicativas se remuestrean a $1^{\circ}$ (resolución 
    nativa de GRACE) y además a la resolución que se busca estimar o resolución objetivo (en este caso $0\text{.}25^{\circ}$), mediante 
    cierta interpolación, ya sea lineal, cúbica, etc.

    \item Se instancia el modelo \textit{Random Forest} regresivo. Con las variables explicativas a 1$^{\circ}$ se entrena y se valida
    con las observaciones de GRACE midiendo el error cuadrático medio y el error medio absoluto.
    
    \item Se generan las diferencias entre las predicciones del modelo y los valores reales para obtener
    datos llamados \textit{residuales} para posteriormente interpolarlos a la resolución objetivo.

    \item Con las 6 variables explicativas remuestradas a la resolución objetivo se predice el valor de \textit{TWS}
    y posteriormente se adiciona la data residual interpolada a la predicción, esto como una estrategia para introducir entropía.
    
    \item Finalmente, se substraen los valores de algunas variables explicativas para obtener el 
    \textit{Ground Water Storage}, o almacén de agua subterránea en cada punto de la grilla y validar estos valores con
    mediciones en terreno de pozos subterráneos u \textit{observations wells}.
    
\end{itemize}

%
%
%
%
\section{Recolección y preprocesamiento de datos}


La primera parte del trabajo consistió en la recolección de datos necesarios para replicar la metodología
propuesta en la sección 3.1, para esto se recurren a diferentes fuentes de data que nos otorguen observaciones igual
o de mejor calidad que las que se proponen.

En base a las variables explicativas presentes en \cite{11}, se proceden a investigar múltiples fuentes de datos, las cuales difieren de las utilizadas en el paper de referencia. 
Las bases de datos seleccionadas fueron \textbf{(CR)$^2$-met} para los datos de precipitación, \textbf{ERA5-Land} para las 5 variables explicativas restantes,
observaciones de pozos provienientes de la \textbf{DGA} y finalmente los datos de GRACE obtenidos de \textbf{OPeNDAP}.

El período de tiempo que considera el estudio comienza el mes de abril del año 2002 y culmina en febrero del año 2017 en instantes de tiempo en los cuales existan mediciones en cada una
de las variables.

En cuanto a los límites espaciales, el estudio es realizado en el parche que tiene como límites longitudinales desde 77.5°E hasta 66.5°E y latitudinales desde 16.5°S hasta
56.75°S, el cual corresponde a todo el territorio continental chileno.

    \subsection{GRACE}
    Los datos satelitales georrefereenciados de GRACE se encuentran empaquetados mensualmente durante todo el período de estudio
    para cualquier coordenada geográfica del globo. Por tanto el primer desafío ocurre al momento de extraer los datos de interés que son los  
    del territorio chileno, el cual resulta ser un proceso heurístico exhaustivo debido a la configuración de coordenadas (latitud, longitud) del sitio web,
    ya que la longitud se encontraba entre el intervalo $[0,180]$ cuando normalmente esta se encuentra entre los valores $[-90,90]$. Luego de múltiples intentos
    se consigue dar con las coordenadas que localizan al territorio como se muestra en la figura 3.2(a).
    
    \begin{figure}[H]
        \centering
              \subfigure[Datos de longitud y latitud para cada variable]{\includegraphics[height=3cm]{images/GRACE_webpage.png}\label{fig:vt:a}}
              \subfigure[Gráfico de GRACE]{\includegraphics[height=4.6cm]{images/GRACE_init.png }\label{fig:vt:b}}\goodgap
              \vskip -0.1in
        \caption[Rutina para extracción de datos GRACE]{Gráfico preliminar de datos de GRACE para las coordenadas de Chile continental en mes particular.}
        \label{logo}
    \end{figure}

    En la figura se aprecia que existen dos variables involucradas en el conjunto de datos: \textit{lwe\_thickness}, que es la abreviación del inglés
    \textit{level water equivalent thickness}, que se traduce como grosor de agua equivalente y \textit{uncertainty} la cual es la incertidumbre
    o el grado de error sobre la predicción del modelo. Las variables mencionadas poseen 3 características: tiempo, latitud y longitud.

    Entrando en el análisis cuantitativo, se puede graficar la distribución de datos de GRACE a baja resolución:
    \begin{figure}[h]
        \centering
              \includegraphics[scale=0.6]{images/distGRACEnoproc-1deg.png}
        \caption[Distribución de datos satelitales a baja resolución]{\footnotesize Distribución de datos satelitales GRACE sin procesar a baja resolución.}
    \end{figure}
    
    Claramente se percibe una distribución muy semejante a la normal, los valores de la media son 0.002110 y una desviación estándar de 0.05371, con valores extremales -0.375 como mínimo y 
    máximo de 0.484. Por ende, no es poible determinar el mismo criterio que en la figura \ref{logo1} ya que la escala de la zona de estudio es mucho menor, por esta razón, es mejor convenir las siguientes clasificaciones
    de acuerdo al valor de la anomalía:

    \begin{table}[H] 
        \caption[Rangos de valores de la anomalía TWS en Chile]{Representaciones del valor de la anomalía TWS del satélite GRACE para todo su dominio.}
        \newcolumntype{C}{>{\centering\arraybackslash}X}
        \begin{tabularx}{0.9\textwidth}{CCC}
        \toprule
        \textbf{Intervalo}	& \textbf{Categoría}	&\textbf{Método de cálculo}\\
            \midrule
            \textbf{$\left[-0\text{.}375,~-0\text{.}057\right[$}    & Más seco que el periodo anterior.       & $\left[v_{min},~ \mu - \sigma\right[$\\
            \textbf{$[-0\text{.}057,~0\text{.}0618]$}	            & Cercano a condiciones normales.         & $[\mu - \sigma,~ \mu +\sigma]$ \\
            \textbf{$\left]0\text{.}0618,~0\text{.}484\right] $}    & Más húmedo que el perídodo anterior.    & $\left]\mu +\sigma,~v_{max}\right]$\\
            \bottomrule
        \end{tabularx}
    \end{table}
    
    \subsection{ERA5-Land}
    ERA5-Land es un conjunto de datos que proporciona una visión coherente de la evolución de las variables terrestres durante varias décadas 
    con una resolución mejorada. ERA5-Land se ha producido reproduciendo el componente terrestre del reanálisis climático ECMWF ERA5. El reanálisis combina 
    datos de modelos con observaciones de todo el mundo en un conjunto de datos globalmente completo y consistente utilizando las leyes de la física. 
    Además, produce datos que se remontan varias décadas atrás en el tiempo, proporcionando una descripción precisa del clima del pasado \cite{ERA5-L}.

    El conjunto de datos son del tipo cuadriculado, con cuadrícula regular de latitud y longitud a una resolución de 0.1$^{\circ} (\sim 9$ Km), 
    para el proyecto en cuestión, de todo el territorio continental chileno.

    Las variables utilizadas como explicativas del modelo \textit{Random Forest} fueron la humedad del suelo (\textit{soil moisture}), nieve equivalente en agua (\textit{snow water equivalent}),
    evaporación por transpiración de la vegetación (\textit{evaporation from vegetation transpiration}), agua de dosel (\textit{canopy water}) y escorrentía (\textit{runoff}).
    
    \subsubsection{Evaporación por transpiración de la vegetación}
    Cantidad de evaporación de la transpiración de la vegetación. Esto tiene el mismo significado que la extracción de raíces, 
    es decir, la cantidad de agua extraída de las diferentes capas del suelo. 
    Esta variable se acumula desde el comienzo del tiempo de pronóstico hasta el final del paso de pronóstico y está medida en metros de agua equivalente \cite{ERA5-L}.

    \subsubsection{Escorrentía}
    Parte del agua de la lluvia, de la nieve que se derrite o de lo profundo del suelo, permanece almacenada en las capas subterráneas. De lo contrario, el agua se escurre, ya sea sobre la superficie (escorrentía superficial)
    o bajo tierra (escorrentía subterránea) y la suma de estos dos se denomina simplemente ``escorrentía''. Las unidad de medida de la escorrentía es la profundidad en metros. Esta es la profundidad que tendría el agua si se distribuyera uniformemente sobre el píxel de la rejilla.
    La escorrentía es una medida de la disponibilidad de agua en el suelo y puede, por ejemplo, utilizarse como indicador de sequía o inundación.

    \subsubsection{Nieve equivalente en agua}
    La profundidad que presenta la nieve en su área de cuadrícula se puede medir en metros de agua equivalente, por lo que coincide con la altura que tendría el agua líquida acumulada si la nieve se derritiera y se 
    repartiera uniformemente en cada pixel. El Sistema de Pronóstico Integrado ECMWF representa la nieve como una sola capa adicional sobre el nivel superior del suelo. La nieve puede cubrir todo o parte de la caja de rejilla.

    \subsubsection{Agua de dosel}
    El dosel (\textit{canopy}), consiste en el área formada por las copas de los árboles en un bosque. Por lo tanto, el agua de dosel consiste en el agua presente en cada una de las hojas de los árboles, dicha agua proviene de 
    la precipitación mayoritariamente, pero también se ve afectada por la humedad, la temperatura, entre otros factores.

    \subsubsection{Humedad del suelo}
    Esta variable explicativa consiste en la suma de 4 parámetros hidrológicos, éstos se definen como el volumen de agua presente en cada capa de suelo. La primera de ellas
    hace referencia al volumen de agua presente en la superficie (0 m) hasta los 0.07 m. La segunda capa parte en los 0.07 m y culmina en los 0.28 m. La tercera capa llega hasta el metro de profundidad partiendo desde los 0.28 m
    y la capa final, desde 1 metro hasta los 2.89 metros \cite{ERA5-L_doc}. Con estos parámetros se genera la variable \textit{soil moisture}.

    \subsection{Precipitación}
    El conjunto de datos CR2-MET contiene información meteorológica (precipitación, temperaturas medias y extremas) en un grilla rectangular de 0.05º latitud-longitud (aproximadamente 5 Km) para el territorio de Chile continental 
    durante el periodo de 1979-2016. La técnica utilizada para la construcción del producto de precipitación se basa en una regionalización estadística de datos del reanálisis atmosférico ERA-Interim (datos disponibles en grillas de $\sim$70 Km). 
    El método utiliza modelos estadísticos como funciones de transferencia para traducir precipitación, flujos de humedad y otras variables \cite{pr}.
    
    \subsection{Observaciones de pozos}
    Las observaciones de pozos de bombeo fueron otorgadas por 2 bases de datos diferentes de la DGA. Ambas bases de datos constaban de una serie de variables útiles para muchos estudios.
    Particularmente para el trabajo se utilizaron los códigos llamados \textbf{ESTID} \cite{gwl}, los cuales constan de códigos únicos que poseen una coordenada geográfica asociada localizada dentro
    del territorio chileno, la \textbf{fecha de la medición}, las cuales se encontraban distanciadas diariamente desde enero del 2000, hasta enero del año 2016 con valores faltantes y finalmente el valor de la \textbf{profundidad del pozo},
    la cual se encontraba en metros y representa la distancia desde la superficie terrestre hasta la superficie acuática de la cuenca, es decir, mientras mayor sea la prfundidad de un pozo, menos contenido de agua este presenta.

    Ambas bases de datos debieron corroborarse ya que estas poseían códigos ESTID comunes y aquellos que diferían en medición se optaba por descartarse debido a la inconsistencia presente.
    Finalmente el número de códigos que fueron sujeto a estudio fue de 1036. 

    \begin{figure}[H]
        \centering
              \includegraphics[scale=0.4]{images/Observacion_pozos.jpeg}
        \caption[Datos de Pozos DGA sin procesar]{\footnotesize Gráfica del valor promedio de cada código único para todo el intervalo de tiempo.}
        \label{pozosDGA}
    \end{figure}

    El objetivo de estos datos recae en validar el contenido de agua terrestre (TWS) a una alta resolución, es decir, corroborar las predicciones del modelo mediante el estudio de tendencias y estacionalidades a través de series de tiempo. 
    Como se muestra en la figura \ref{pozosDGA}, los datos provenientes de la DGA consideran solamente la zona norte y centro del país por lo que el estudio posterior de validación del
    \textit{downscalling} deberá estudidarse dentro de alguna de las zonas en las que existan mediciones en terreno.

    La primera intervención realizada a esta nueva base de datos unificada consistió en relacionar estos datos que se encuentran repartidos de manera no uniforme en el territorio. Es por esto que luego de aplicarle el remuestreo temporal detallado más adelante, el siguiente paso fue transformar los datos con un código ESTID asociado,
    a datos grillados. Donde las grillas serán la resolución original (1$^{\circ}$) y la grilla de alta resolución (0.25$^{\circ}$). Es claro que habrán pixeles de la grilla en donde 
    se encuentren más de una estación de medición y en otras donde no existan observaciones, es por esto que se debe adicionar nuevas variables, tales como \textbf{la densidad de pozos}, que se define como la cantidad de pozos presentes en cada pixel y además
    la \textbf{cantidad de observaciones}, es decir, la cantidad de datos presentes en cada punto de la grilla, esto para no perder aquellos píxeles que sean más representativos o que 
    contengan una alta riqueza en sus mediciones.

    \subsection{Remuestreo espacial}
    
    El remuestreo espacial fue una de las tareas más importantes a realizar durante el preprocesamiento. Este consiste en que a partir de los datos a cierta
    resolución nativa u original se buscan aumentar o disminuir dicha resolución. Para el rescalamiento espacial de los datos georreferenciados se optó por utilizar una herrmienta que sea capaz de convertir 
    cuadrículas de datos a la resolución deseada de forma rápida y que pueda operar con grillas predefinidas según las necesidades, por lo tanto CDO es el software indicado. 

    
    \begin{figure}[H]
        \centering
              \subfigure[Densidad a 0.25$^{\circ}$]{\includegraphics[scale=0.4]{images/wells_density025.png}}
              \subfigure[Densidad a 1$^{\circ}$]{\includegraphics[scale=0.4]{images/wells_density1.png }}\goodgap
              \vskip -0.1in
        \caption[Cantidad de pozos en cada grilla]{Gráfico de cantidad de pozos por píxel para grilla nativa y de alta resolución para el mes de enero del año 2005.}
        \label{pixeldepth}
    \end{figure}

    
    El rescalamiento espacial se realiza sobre datos con una alta resolución espacial, en este caso las variables explicativas de ERA5-L y precipitación, luego, mediante interpolación bilinear a través de una rutina en \textit{bash} de CDO se consigue
    la resolución objetivo (0.25$^\circ$) de\textit{Downscaling} y la resolución original de GRACE (1$^\circ$). Es claro que hay muchas maneras de formar las grillas de datos los cuales se encuentren a una resolución de 
    1$^\circ$ y que éstas no necesariamente tienen que ser las mismas entre ellas, es por esto que se fija la grilla a baja resolución y alta resolución antes de realizar el remuestreo
    para que todos los datos se encuentren remuestrados por una misma grilla. Lo más sencillo es extraer los datos de GRACE y de ellos obtener las coordenadas georreferenciadas a resolución de 1$^\circ$ y así se disminuye la cantidad de
    variables a remuestrar.

    Aunque CDO entregue resultados mucho más rapido que Pyhon, la mayor desventaja recae en el hecho que el software es ajeno al código en el cual se han realizado todos y cada uno de los análisis anteriores, por tanto se deberían realizar otros procedimientos para poder integrar las operaciones a los archivos resultantes.
    A partir de esto, se opta por realizar interpolación bilinear de forma interna en Python y, para corroborar los resultados, se recurre a CDO que al ser un software especializado nos entrega seguridad a la hora de realizar el remuestreo
    espacial. Se escoge la variable hidrológica \textit{runoff} proveniente del conjunto de datos ERA5-L y se procede a realizar las interpolaciones para el mes de marzo del año 2009 y se obtuvieron los siguientes resultados:
    
    \begin{table}[H] 
        \caption[Comparación de interpolaciones Python/CDO]{Comparación de interpolación bilinear realizada en software especializado (CDO) versus Python desde la resolución original de 0.1° a 0.25°.}
        \newcolumntype{C}{>{\centering\arraybackslash}X}
        \begin{tabularx}{0.9\textwidth }{CCCC}
        \toprule
         \textbf{Datos}& \textbf{Intervalo}	&\textbf{Media ($\mu$)}  &\textbf{Desviación estándar ($\sigma$)}\\
            \midrule
            \textbf{Originales 0.1°}	&	[0, 0.01130]       &  0.0002417  & 0.000788\\
            \textbf{CDO 0.25°}	        &   [0, 0.01058]       &  0.0002242  & 0.000755\\
            \textbf{Python 0.25°}       &   [0.00024, 0.00083] &  0.0002430  & 0.000831\\
            \bottomrule
        \end{tabularx}
    \end{table}

    Si bien las cotas de los datos para Python se ven mucho más restringidas, este análisis no nos entrega información de qué tan cercanas son cada una de las interpolaciones entre sí. Claramente CDO será el punto de comparación debido a que
    la función para la cual el software fue diseñado, es operar en grillas de datos, el objetivo ahora es calcular qué tan lejos se encuentra la interpolación de Python con la de CDO.
    Así, se proceden a calcular los errores ya conocidos, \textit{MAE} y \textit{RMSE}. Obteniendo un valor de error medio absoluto de $1.93317\times 10^{-6}$ y un error cuadrático medio de $5.262181\times 10^{-10}$. Por lo tanto, 
    la interpolación realizada en Python resulta ser válida en base a la interpolación que se realiza utilizando CDO.


    \subsection{Remuestreo temporal}
    Debido a la naturaleza de los datos de precipitación y de mediciones de pozos de bombeo, fue necesario realizar un preprocesamiento 
    adicional al de ERA5-L ya que además del remuestreo espacial a 1$^\circ$ y 0.25$^\circ$ se debió realizar un remuestreo temporal
    mensual para adicionar la variable al modelo.

    Junto con la alta riqueza espacio-temporal de la base de los productos grillados de precipitación surge un problema no menor: la capacidad de procesamiento para los datos aumenta
    exponencialmente, por este motivo se busca realizar el procesamiento temporal mediante la librería \texttt{xarray} de Python y no en \texttt{pandas} como se había hecho con
    ERA5, esto para dar paso al procesamiento espacial el cual se debió realizar mediante la librería de \texttt{geopandas} debido a la naturaleza misma de los datos
    los cuales se encuentran georreferenciados.

    Debido a que la naturaleza de las observaciones de pozos, el problema de capacidad de cómputo desaparece y no resulta problema remuestrear mensualmente el valor de la profundidad a través de 
    un promedio simple para cada código de medición.
%   
%
%
%
%
%
%
\section{Entrenamiento y Testeo}
    Para entrenar el modelo Random Forest, en primera instancia es necesario dividir el conjunto de datos en entrenamiento
    y testeo, para luego verificar la predicción en base a observaciones que no han sido utilizadas en el entrenamiento. Debido a que las variables explicativas son hidrológicas y 
    el territorio nacional presenta diversos climas en toda su extensión, surge la idea de estratificar los datos de entrenamiento y testeo mediante las macrozonas del país para conseguir predicciones coherentes 
    y que la división de conjuntos no se encuentre sesgada de ninguna manera.
    \begin{figure}[H]
        \centering
              \includegraphics[width=\textwidth]{images/division_zonas.png}
              \vskip -0.1in
        \caption[División de datos por macrozonas de Chile]{\footnotesize División de datos en base a las zonas de Chile.}
        \label{division}
    \end{figure}

    \begin{figure}[H]
        \centering
            \includegraphics[width=\textwidth]{images/traintest.png}
              \vskip -0.1in
        \caption[Entrenamiento y testeo mediante macrozonas]{\footnotesize División en entrenamiento y testeo para cada macrozona de Chile.}
        \label{traintest}
    \end{figure}

    En la figura \ref{division}, \texttt{DATA} constituye todas las variables explicativas para el modelo con mediciones remuestradas en la grilla de GRACE
    a 1$^{\circ}$, el valor de la variable respuesta (\texttt{lwe\_thickness}), además de la respectiva fecha de medición (\texttt{time}) y 
    coordenadas geográficas (\texttt{lon} y \texttt{lat})
%
%
%
%
\section{Construcción del modelo Random Forest}
Según el sitio web de sklearn \cite{rfskl}, un bosque aleatorio es un metaestimador 
que ajusta una serie de árboles de decisión en varias submuestras 
del conjunto de datos y utiliza el promedio para mejorar la precisión predictiva y controlar el sobreajuste. 
El tamaño de la submuestra se controla con el parámetro \texttt{max\_samples}, si el parámetro \texttt{bootstrap=True}, el cual se encuentra predeterminado, 
de lo contrario, se usa todo el conjunto de datos para construir cada árbol.

Para la construcción del modelo se crearon 2 predictores, uno de ellos minimiza el error cuadrático medio (fig. \ref{logo2}) y el otro minimiza el error medio absoluto (fig. \ref{logo3}). En cuanto a la construcción misma, ambos modelos comparten los mismos hiperparámetros, 
éstos son:
\begin{itemize}
    \item \texttt{n\_estimators}: Es la cantidad de árboles presentes en el \textit{ensemble}. 
    \item \texttt{max\_depth}: Se define como la profundidad del árbol, es decir, la cantidad máxima de particiones de cada estimador.
    \item \texttt{max\_features}: Es la cantidad de características que se consideran para encontrar la mejor división del árbol, o en otras palabras, el porcentaje de variables explicativas utilizadas para realizar la mejor división en cada árbol. \texttt{None} 
    se refiere a que el algoritmo seleccione la profundidad del árbol de acuerdo a la minimización de su función objetivo.
    \item \texttt{oob\_score}: La abreviación para \textit{out of bag score} definida como el porcentaje de observaciones las cuales no serán utilizadas para entrenar el modelo a la hora de hacer la muestra de \textit{boostrap} y por tanto solo serán utilizadas para testear.
    \texttt{False} significa que el modelo utilizará todo el conjunto para realizar el \textit{bagging}.
    \item \texttt{n\_jobs}: Hace alusión a la cantidad de procesadores virtuales que se utilizarán para entrenar el modelo gracias a que el \textit{bagging} es fácilmente paralelizable y en Python se encuentra implementada esta función, en este caso se le asigna el valor de  
    $-1$ para que se utilicen todos los procesadores disponibles y así disminuir el tiempo de cómputo.
    \item \texttt{random\_state}: Define la semilla con la cual se generarán las muestras aleatorias, por lo que si se le otorga un valor se puede obtener un modelo replicable.
\end{itemize}

\begin{figure}[H]
    \centering
          \includegraphics[width=\textwidth]{images/RF_regressor.png}
          \vskip -0.1in
    \caption[Construcción y entrenamiento del modelo Random Forest con criterio de \textit{mse}]{\footnotesize Modelo Random Forest construído en Python con minimización del error cuadrático medio.}
    \label{logo2}
\end{figure}

\begin{figure}[H]
    \centering
          \includegraphics[width=\textwidth]{images/RF_regressor_mae.png}
          \vskip -0.1in
    \caption[Construcción y entrenamiento del modelo Random Forest con criterio de \textit{mae}]{\footnotesize Modelo Random Forest construído en Python con minimización del error medio absoluto.}
    \label{logo3}
\end{figure}

    \subsection{Interpolación de Residuos}

    Posterior a instanciar cada uno los modelos y seleccionar el que mejor métricas de validación entregue en la resolución nativa, corresponde calcular la diferencia entre los valores reales y los valores predichos, es decir, los datos residuales.
    Luego de obtener dichos datos es necesario remuestrearlos a la resolución objetivo, esto mediante interpolación bilinear.
 
    Mediante dos funciones creadas en Python y la clase \texttt{interpolate.interp2d} de la librería \texttt{scipy} es posible interpolar los datos a partir de sus 4 vecinos más cercanos resolviendo el sistema de ecuaciones
    de la ecuación \ref{interp2d}, estas funciones son:
    \begin{itemize}
        \item {\texttt{nearest\_coor\_patch}: Es una función que particiona la grilla de longitud (\texttt{df\_lon}) y de latitud (\texttt{df\_lat}) del \textit{dataframe} y retorna los 4 vecinos más cercanos en base a la distancia euclídea
        de las coordenadas de longitud y latitud entregadas.}
        \item {\texttt{bilinear\_interpolation}: Función que interpola el parche de coordenadas entregado (\texttt{patch}) de la variable \texttt{var\_name} en la longitud (\texttt{new\_lon}) y latitud (\texttt{new\_lat}) dadas. En el caso de haber un error cuantitativo de interpolación,
         la función retorna el objeto nulo de la librería \texttt{numpy}.}
    \end{itemize}
    \begin{figure}[H]
        \centering
              \includegraphics[width=\textwidth]{images/res_interp.png}
              \vskip -0.1in
        \caption[Funciones utilizadas para la interpolación de residuos]{\footnotesize Funciones utilizadas para interpolar bilinearmente la data residual de 1$^{\circ}$ a 0.25$^{\circ}$.}
        \label{logo4}
    \end{figure}

    \begin{figure}[H]
        \centering
              \includegraphics[width=\textwidth]{images/res_interp_rutine.png}
              \vskip -0.1in
        \caption[Rutina de interpolación de residuos]{\footnotesize Rutina para interpolar bilinearmente la data residual de 1$^{\circ}$ a 0.25$^{\circ}$.}
        \label{logo5}
    \end{figure}

    Posteriormente, se toman las grillas mensuales y para cada coordenada en la grilla a baja resolución se procede a interpolar en base a la funciones antes mencionadas. La rutina de interpolación de residuos se detalla en la figura \ref{logo5} para almacenar los valores residuales en la variable \texttt{data\_res}.
    
    Finalmente, los datos residuales interpolados son añadidos a la predicción a alta resolución como una forma de agregar entropía al modelo y así otorgarle un grado de intertidumbre al valor final.



   