\chapter{Conclusiones}
\label{C6}

La metodología del \textit{Downscaling} propuesta por \cite{11} aunque no fue implementada de inicio a fin, logra entregar buenos resultados a pesar de no poseer un método de validación ortodoxo en la alta resolución, lo cual
complejiza más el trabajo. Aunque la validación tuvo que constar de varias etapas, todas en conjunto presentan las bases de la distribución obtenida con el modelo predictivo.

La extensión de la librería de \texttt{pandas} para el tratamiento y estudio de datos georreferenciados resultó ser una herramienta indispensable para el aumento de resolución, sin la ayuda de \texttt{geopandas} habría sido
mucho más complejo y tardío llebar a cabo el proyecto en cuestión.

La selección, análisis y pre-procesamiento de datos son sin lugar a dudas las tareas más importantes y las que requieren más tiempo, si bien no se escogieron los mismos conjuntos de datos que en la métodología que se discute 
en el capítulo 3, sin lugar a duda los datos que se consiguieron para el territorio chileno además de ser homogéneos, son fiables ya que son proporcionados por centros de climatología especialzados y/o instituciones nacionales, lo cual
es una gran ventaja.

La validación, además de ser una de las tareas más complejas de este proyecto, fue una de las más importantes, ya que se buscaba que las tendencias y comportamientos a gran escala se mantengan a pequeña escala. Las observaciones de pozos
logran corroborar que las tendencias de GRACE a baja y alta resolución coinciden, y aquellas zonas en las que no, los comportamientos poco similares sean causados posiblemente por la explotación de los recursos hídricos o por la adición de variabilidad por parte de las
variables explicativas.

Con los productos grillados obtenidos es posible ampliar aún mas el estudio de las aguas subterráneas en Chile, teniendo un producto más granular los estudios en zonas áridas y zonas que presenten megasequía serán 
más compresibles y particulares, al igual que en las zonas donde aumente el contenido hídrico.

Como es apreciable en este trabajo, en Chile existe un déficit hídrico que cada año aumenta, sobre todo en la zona centro-sur del país. Debemos ser conscientes de lo que está sucediendo hoy, pues son irreparables las consecuencias que trae
la extracción excesiva de las cuencas del territorio y por lo que se aprecia, serán nuestras fuentes de agua más favorables.