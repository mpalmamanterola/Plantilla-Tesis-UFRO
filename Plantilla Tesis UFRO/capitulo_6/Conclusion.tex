\chapter{Conclusiones}
\label{C6}
Los modelos basados en árboles resultan ser algortimos que logran realizar la tarea de reducción de dimensión con un error pequeño gracias a la partición característica
del espacio que éstos logran.

Random Forest quizá no sea el modelo predictivo más contemporaneo, es por esta razón que las métricas en series de tiempo y de validación 
en el conjunto de testeo puedan mejorarse con un modelo basado en árboles más actual, como por ejemplo \texttt{XGBoost} el cual, además de agustar árboles
también ajusta los hiperparámetros de forma óptima a través de múltiples herramientas, como la \textbf{paralelización}, el \textit{\textbf{tree pruning}} e inclusive
la \textbf{optimización de hardware}.