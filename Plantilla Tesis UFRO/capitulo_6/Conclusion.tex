\chapter{Conclusiones}
\label{C6}

La metodología del \textit{Downscalling} aunque no haya sido implementada de inicio a fin logra entregar buenos resultados a pesar de no poseer un método de validación ortodoxo en la alta resolución, lo cual
complejiza más el trabajo.

La extensión de la librería de \texttt{pandas} para el tratamiento y estudio de datos georreferenciados resultó ser una herramienta indispensable para el aumento de resolución, sin la ayuda de Geopandas habría sido
mucho más complejo y tardío llebar a cabo el proyecto en cuestión.

La selección, análisis y pre-procesamiento de datos son sin lugar a dudas las tareas más importantes y las que requieren más tiempo, si bien no se escogieron los mismos conjuntos de datos que en la métodología que se discute 
en el capítulo 3, sin ligar a duda los datos que se consiguieron para el territorio chileno además de ser homogéneos, son fiables ya que son datos proporcionados por centros de climatología espcialzados y/o instituciones nacionales, lo cual
es una gran ventaja.

Random Forest quizá no sea el modelo predictivo más contemporaneo, si bien el modelo otorga buena generalización es posible mejorar aún más los resultados con un modelo basado en árboles más actual, como por ejemplo \texttt{XGBoost} el cual, además de agustar árboles
también ajusta los hiperparámetros de forma óptima a través de múltiples herramientas, como la \textbf{paralelización}, el \textit{\textbf{tree pruning}} e inclusive
la \textbf{optimización de hardware}.

La validación, además de ser una de las tareas más complejas de este proyecto, fue una de las más importantes, ya que se buscaba que las tendencias y comportamientos a gran escala se mantengan a pequeña escala. Las observaciones de pozos
logran corroborar que las tendencias de GRACE a baja y alta resolución coinciden, y aquellas zonas en las que no, los comportamientos poco similares sean causados posiblemente por la explotación de los recursos hídricos.
