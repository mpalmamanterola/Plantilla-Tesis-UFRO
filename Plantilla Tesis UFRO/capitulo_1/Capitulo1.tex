\chapter{Introducción}

\section{Contexto y descripción del problema}
Las aguas subterráneas tienen una relación muy cercana con la sociedad humana gracias a que almacenan 
agua de buena calidad en grandes cantidades, se encuentran ampliamente distribuidas, poseen estabilidad 
y facilidad de uso por lo que se ha convertido en una fuente de suministro ideal y está siendo altamente 
utilizada por las personas \cite{11} y mayoritariamente por industrias de manera contínua y persistente sin 
fiscalización alguna. La sobreexplotación de los almacenes de agua subterránea ha generado que el nivel 
continúe disminuyendo, provocando desastres como hundimientos de suelo, pérdida de volumen a través de la
 separación de un acuífero y el secado de zonas húmedas.
Las cuencas son monitoreadas en terreno a través de sondas, este estudio además de ser costoso requiere 
una distribución uniforme lo suficientemente distanciada entre cada estación de medición para captar 
detalles de los cambios de nivel de los acuíferos. 
Con el surgimiento de algunas nuevas tecnologías de observación de la tierra, 
se ha proporcionado un nuevo método de detección remota para el estudio de las variaciones 
del almacenamiento de agua a nivel espacial, éste es el satélite GRACE, el cual a través de 
diferencias gravitacionales logra estimar la cantidad de agua presente en un punto geográfico. 
Muchos investigadores han demostrado que el satélite GRACE puede monitorear bien los cambios en áreas 
a gran escala con un error mínimo \cite{11}. Debido a que el satélite proporciona datos a una baja resolución es 
necesario realizar procesamiento de datos llamado reducción de escala que convierte datos de gran escala
y baja resolución en datos de pequeña escala a una alta resolución. 
%
%
%
%
\section{Satélite GRACE}
La misón GRACE(por sus siglas en inglés \textit{Gravity Recovery and Climate Experiment}), lanzada el 17 de marzo del año 2002
bajo el Programa Pathfinder de Ciencias del Sistema Terrestre de la NASA \cite{11}, consiste en dos satélites 
que orbitan sobre la Tierra a una altitud aproximadamente de 500 Km y a una distancia entre ellos de 200 Km\cite{tws} para así detectar variaciones gravitacionales que resultan
de los movimientos de las masas, en particular de agua en cualquiera de sus estados.
Gracias a este par de satélites es posible detectar las variaciones de la fuerza de gravedad 
mediante el traspaso de información del primer satélite hacia el segundo captando la diferencia de 
masas que existen en el terreno. Debido a estas diferencias gravitatorias se provoca un aumento o 
disminución de la distancia relativa entre un satélite y el otro logrando captar diferencias de hasta 
$1\mu m$, estas mediciones se conocen como \textit{TerrestriaL Water Storage(TWS)}
El gran objetivo de GRACE, es poder captar el movimiento de las aguas alrededor del globo para 
contrarrestar riesgos como sequías, inundaciones, desprendimiento de suelos o socavones.

\subsection{Almacenamiento de agua terrestre}
El Almacenamiento de agua terrestre(en inglés \textit{TWS}) es definido como una estimación integrada del agua almacenada sobre y debajo de la superficie terrestre, 
incluyendo el agua de dosel(\textit{canopy water}), ríos y lagos, humedad de suelo(\textit{soil moisture}) y aguas subterráneas \cite{tws}. Es calculado de la siguiente manera:
\begin{equation}
    \text{TWS Anomaly}_t = \frac{TWS_t-\bar{X}}{\delta}
\end{equation}
donde $TWS_t$ es el valor de TWS en el mes $t$, $\bar{X}$ es el valor promedio a largo plazo de TWS y $\delta$ es su desviación estándar. Ambas calculadas para el mismo mes $t$ utilizando 
los datos temporales disponibles\cite{tws}.

En base a las anomalías se pueden definir ciertos rangos de comportamientos de las mediciones que se detallan en la figura \ref{logo1}, en ella se puede observar que para la anomalía TWS, si ésta se encuentra en el intervalo $[-2,-1[$
implica que  el contenido terrestre de agua está más seco de lo normal. Por el contrario si la observación se encuentra en el intervalo $[1,2[$ representa que el contenido de agua ha aumentado y finalmente si se encuentra entre
$[-1,1[$ representa ``normalidad'' en el contenido de agua, es decir, la cantidad de agua presente no ha variado considerablemente.

\begin{figure}[H]
    \centering
          \subfigure[Zonas con estrés hídrico en base a TWS]{\includegraphics[scale=0.25]{images/tws-example.png}\label{fig:tws}}
          \subfigure[Descripción general del intervalo donde oscila TWS]{\includegraphics[scale=0.35]{images/tws-range.png }\label{fig:tws1}}\goodgap
          \vskip -0.1in
    \caption[Ejemplo de anomalías de TWS]{\footnotesize Ejemplo de anomalías de TWS donde se destacan las condiciones de estrés hídrico en Australia 2019}
    \label{logo1}
\end{figure}
Este indicador juega un  papel crucial en el ciclo hidrológico mundial y el proceso de integración tierra-atmósfera. Las variaciones en \textit{TWS} son
un fuerte indicador del equilibrio o desequilibrio de los flujos de agua afectados por las condiciones climáticas o geográficas
regionales y es uno de los factores importantes que vinculan la circulación general de la atmósfera con los desastres naturales(por ejemplo, la sequía)\cite{15}.
%
%
%
%
\section{(CR)$^2$}
El Centro de Ciencia del Clima y la Resiliencia (CR)$^2$ es un centro de excelencia cuyo propósito es generar investigación
sobre la ciencia del clima y la resilencia desde un enfoque interdisciplinario. 

(CR)$^2$ nace en 2013 financiado por Fondap de la Agencia Nacional de Investigación y Desarrollo(Anid). En 2018 comienza una segunda etapa
de investigación donde busca consolidarse como un actor clave para la ciencia del clima y así contribuir al tránsito del país hacia un desarrollo 
bajo en carbono\cite{cr2}.

%
%
%
%
\section{Dirección General de Aguas}
La Dirección General de Aguas es el organismo del Estado de Chile que se encarga de gestionar, verificar y difundir la infromación 
hídrica del país, en especial respecto a su cantidad y calidad, las personas naturales o jurídicas que están autorizadas a utilizarlas,
las obras hidráulicas existentes y la seguridad de las mismas con el objetivo de contribuir a una mayor competitividad 
del mercado y el resguardo de la certeza jurídica e hídrica para el desarrollo sustentable del país\cite{dga}.

%
%
%
%
\section{Climate Data Operator}
El software \textit{Climate Data Operator(CDO)} es una colección de operadores para procesamiento y predicción estándar de datos climatológicos.
Los operadores incluyen funciones estadísticas y artimeticas simples, selección de datos, herramientas de submuestreo e interpolación espacial.
\textit{CDO} fue desarrollado para tener el mismo conjunto de funciones de procesamiento para archivos con extensión GRIB y NetCDF. Las mayores característica
de CDO son, entre otras, los más de 700 operadores disponibles, diseño modular y fácilmente extensible a nuevos operadores, interfaz de linea de comando UNIX muy simple,
un conjunto de datos puede ser procesado mediante varios operadores sin la necesidad de almacenar los resultados provisionales en archivos y lo más importante un rápido procesamiento
de conjuntos de datos muy grandes. \cite{16}.
%\newpage
%
%
%
%
\section{Objetivo General}

Implementar \textit{Downscalling} a los datos satelitales GRACE a través de un modelo predictivo para obtener 
productos grillados de \textit{Terrestrial Water Storage} en Chile a una alta resolución.
%
%
%
%
\section{Objetivos Específicos}
\begin{itemize}
    \item Definir la metodología del \textit{Downscalling}.
    \item Adquirir conocimientos acerca de tratamiento de datos satelitales georreferenciados utilizando Python.
    \item Recolectar datos necesarios para implementar el aumento de resolución en el territorio.
    \item Analizar y Pre-procesar los datos. 
    \item Implementar el modelo predictivo y medir su rendimiento en base a métricas definidas.
    \item Validar las predicciones del modelo en base a observaciones \textit{in situ} a través de series de tiempo.
\end{itemize}